\section{RationalGRL Methodology and Metamodel}
\label{sect:overview}

In this section we present a high-level overview of the RationalGRL framework. In the first subsection, we present a methodology, specifying how practitioners can use RationalGRL to create goal models with traceability links to underlying arguments. In the second subsection, we link the new argumentation elements and relations to the existing elements and relations of GRL in a metamodel. This metamodel serves as the specification for an implementation. In Section~\ref{sect:tool} we briefly discuss our prototype implementation based on this metamodel.

\subsection{RationalGRL Methodology} 

As mentioned in Section~\ref{sect:introduction}, the RationalGRL framework uses concepts from practical reasoning argument scheme (PRAS) to help integrating goal models with the detailed discussions and arguments the stakeholders pose during the analysis phase. That is, the RationalGRL framework includes two main parts: Argumentation modeling and GRL modeling. 

For the GRL part, we first need to create the ``initial'' GRL model by analyzing the non-functional requirements in the requirements specification document and by refining the high-level goals into operationalized tasks. For the argumentation part, arguments and counterarguments are put forward about various parts of the goal model.
These two parts, GRL and argumentation, are developed iteratively and each side can impact the other side so that the models can be refined or new critical questions and argument schemes can be instantiated. For example, answering a critical question \emph{Is the task \texttt{A} possible?}, instantiated from the argumentation model, can result in removing or adding a task in the GRL model. Similarly,  if, for example, we add a new intentional element to the GRL model, it can lead to a new critical question relevant to this intentional element and its relationships.  Figure~\ref{fig:rationalgrl-framework} presents an overview of RationalGRL framework with its components and their relationships.  The GRL model is shown  on the right-hand side of the framework while the argumentation model is on the left-hand side. The links between the two sides illustrate the impacts and relationships between two sides. Note that answer to critical questions and argument schemes that are instantiated during the analysis phase of the GRL model are documented with the GRL model and can be referred to in the future. 

\begin{figure}[ht]
\centering
\includegraphics[scale=0.4]{img/framework}
\caption{The RationalGRL Framework\todo{F}{M,S}{Can you make this a vector image?}}
\label{fig:rationalgrl-framework}
\end{figure}

We propose the following methodology (shown in Figure~\ref{fig:rationalgrl-methodology}) to develop an instance of the RationalGRL framework. Here we assume that the initial GRL models have been created based on the requirements specification documents and the discussions of the stakeholders. The rest of the steps are as follows:

\begin{figure*}[ht]
\centering
\includegraphics[scale=0.4]{img/methodology}
\caption{The RationalGRL Methodology\todo{F}{M,S}{Can you make this a vector image?}}
\label{fig:rationalgrl-methodology}
\end{figure*}


\textbf{(1) Instantiate Argument Schemes (AS)} -- In this step, we start from the list of arguments schemes of the argumentation framework. We select an intentional element from the initial GRL model that we want to analyze and we instantiate a relevant argument scheme from the already existing list of argument schemes or by adding a new one. For example, an argument scheme can be "Goal \emph{G} contributes to softgoal \emph{S}". When an argument scheme is instantiated, it corresponds to  an argument for or against part of a goal model.

\textbf{(2) Answer Critical Questions (CQs)} -- After instantiating an argument scheme, we invoke related critical questions to attack the argument with counter-arguments.  Each argument scheme includes one or more critical questions. For example, for the argument scheme, "Goal \emph{G} contributes to softgoal \emph{S}", there are two critical questions as follows:  \emph{Does the goal contribute to the softgoal?} and \emph{Does the goal contribute to some other softgoals?}. 
%It is worth mentioning that, answering a critical question may result in a "conflict" situation. This is out of the scope of our work here.  
%MvZ: removed this sentence because I don't know what you mean with "conflict". I think We should explain it better or leave it out.
When the analyst answers  a critical question, a new argument scheme may be instantiated.  Thus, it is possible to go back and forth between this step and step (1).

\textbf{(3) Decide on Intentional Elements and their Relationships} -- By answering a critical question, one of the four following cases can occur: \textsf{INTRO}, \textsf{DISABLE}, \textsf{REPLACE} or \textsf{ATTACK}.  Any of these cases can  impact the arguments and corresponding GRL intentional elements.  \textsf{INTRO} means that 
a new argument scheme is created. That means, the current argument scheme related to the critical question does not get attacked.  In the case of \textsf{DISABLE}, the intentional element or its related links are disabled or removed from the models. \textsf{REPLACE} introduces a new argument and attacks the original argument at the same time. This means that the original element of the argument scheme is replaced with a new one. \textsf{ATTACK} is a generic counterargument which attacks any argument with another argument when new evidence occurs.  

\textbf{(4) Modify GRL Models} -- In this step, we modify the GRL models based on the situation of step (3). That is, one of the following situation can happen with respect to the initial GRL model: 1) a new intentional element or a new link is introduced; 2) an existing intentional element or an existing link gets disabled (removed) from the model; or 3) an existing intentional element or link is replaced by a new one. This results in a new modified GRL. The new GRL model can then impact the argument schemes and instantiate another argument scheme (Step (1)).   

We can continue these four steps until there is no more intentional element or link to analyze or we reach a satisfactory model. 

Answering a critical question can have four different effects on the original argument and the corresponding GRL element:
\begin{itemize} 
\item \textsf{INTRO}: Introduce a new goal element or relationship with a corresponding argument. This operation does not attack the original argument of the critical question, but rather creates a new argument. In Figure~\ref{fig:transcripts:grl}, each GRL element can be seen as the instantiation of an argument scheme. For instance, the XOR-decomposition from ``Generate Cars'' is an instantiation of AS5 as follows: ``Goal \texttt{Generate cars} XOR-decomposes into tasks \texttt{Keep same cars} and \texttt{Create new cars}''. Suppose now that the modelers that created Figure~\ref{fig:transcripts:grl} would continue their analysis by discussing critical question CQ5b: ``Does the goal \texttt{Generate cars} decompose into other tasks?'', and that they would answer this with ``Yes, namely \texttt{Choose randomly}''. This then results in the introduction of another task with the name ``Choose randomly'', and the XOR-decomposes would go from ``Generate Cars'' into the three tasks \texttt{Keep same cars}, \texttt{Create new cars}, and \texttt{Choose randomly}.'' %SG: Very good example. But should we also show it in the model as a new element?
\item \textsf{DISABLE:} Disable the element or relationship of an argument scheme to which critical questions pertains. This operation does not create a new argument, but it only disables (i.e., attacks) the original one. In Figure~\ref{fig:transcripts:grl}, there are several examples of disabled GRL elements. The task ``Add traffic light'' (top-left in figure) is attacked by answering critical question CQ2: ``Is task Add traffic light possible?'' negatively, resulting in an argument that disables the GRL element. What we also see from Figure~\ref{fig:transcripts:grl} is that actor ``Teacher'' is disabled and thus all the elements that are bound to this actor are disabled as well. Furthermore, disabled task ``Dynamic Simulation'' also disables all incoming and outgoing links with this task.

\item \textsf{REPLACE:} Replace the element of the argument scheme with a new element. In Figure~\ref{fig:transcripts:grl}, task ``Show map editor'' has been replaced various times, and this is shown in the figure as a \emph{refined} element. In this case, participants were discussing the correct naming for this element (CQ13), leading to various replacements of the name. While the previous names are not shown in the figure, they show up in the details pane of the corresponding element. %SG: why generate car has a star but not this one?
\item \textsf{ATTACK:} Attack any argument with an argument that cannot be classified as a critical question. In Figure~\ref{fig:transcripts:grl}, we see one example of such a counter-attack. First, task ``Control car influx per road'' is attack by answering CQ2 (irrelevant task). However, after discussing this, participants found that this was not the case, since the problem description stated that it is important that students can control the simulation manually. Therefore, the argument that attacked the task is attacked by the counter-argument ``Important for students'', which re-enables the task ``Control car influx per road''. We provide a precise semantics for this in Figure~\ref{sect:formalframework}.
\end{itemize}


The list of argument schemes and critical questions that we have obtained from our analysis is shown in Table~\ref{table:argument-schemes}. The first four argument schemes (AS0-AS4) are arguments for an element of a goal model, the next seven (AS5-AS11) are related to relationships, the next two (AS12-AS13) are for intentional elements in general, and the last (Att) is a generic counterargument against any type of argument that has been put forward.


It is important to note that the list of argument schemes we present in this section is not exhaustive. It is an initial list that we have obtained by coding transcripts. However, our framework is fully extensible, meaning that new argument schemes and critical questions can be added depending on the problem domain.

As we have already discussed in Section~\ref{sect:overview}, answering critical questions can have different impacts on the model. The right column in Table~\ref{table:argument-schemes} shows the impact of answering each critical question affirmatively. %SG: what does affirmatively mean?
As mentioned earlier, answering a critical question can create an argument disabling the corresponding GRL element of the attacked argument scheme (\textsf{DISABLE}); it can create an argument introducing a new GRL element (\textsf{INTRO}); it can replace the GRL element corresponding to the original argument (\textsf{REPLACE}), or it can simply attack an argument directly (\textsf{ATTACK}).

\begin{table*}[h]
\centering
\begin{tabularx}{\textwidth}{|l|l|l|X|l|l|}
\hline
\multicolumn{2}{|c|}{\textbf{Argument scheme}} & \multicolumn{2}{c|}{\textbf{Critical Questions}} & \textbf{Effect}\\
\hline
AS0 & Actor $a$ is relevant & CQ0 &Is the actor relevant? & DISABLE\\
\hline
AS1 & Actor $a$ has resource $R$ & CQ1 &Is the resource available? & DISABLE\\
\hline
AS2 & Actor $a$ can perform task $T$ & CQ2 &Is the task possible? & DISABLE\\
\hline
AS3 & Actor $a$ has goal $G$ & CQ3 & Can the desired goal be realized? & DISABLE\\
\hline
AS4 & Actor $a$ has softgoal $S$ & CQ4 & Is the softgoal a legitimate softgoal?& DISABLE\\
\hline
\hline
AS5 & Goal $G$ decomposes into tasks $T_1,\ldots,T_n$ & CQ5a & Does the goal decompose into the tasks?& DISABLE\\
& & CQ5b & Does the goal decompose into any other tasks?& REPLACE\\
\hline
AS6 & Task $T$ contributes to softgoal $S$& CQ6a & Does the task contribute to the softgoal?& DISABLE\\
&& CQ6b & Are there alternative ways of contributing to the same softgoal?& INTRO \\
&& CQ6c & Does the task have a side effect which contribute negatively to some other softgoal?& INTRO\\
&& CQ6d & Does the task contribute to some other softgoal?& INTRO\\
\hline
AS7 & Goal $G$ contributes to softgoal $S$ & CQ7a & Does the goal contribute to the softgoal?& DISABLE\\
&& CQ7b & Does the goal contribute to some other softgoal?& INTRO\\
\hline
AS8 & Resource $R$ contributes to task $T$ & CQ8 & Is the resource required in order to perform the task?& DISABLE\\
\hline
AS9 & Actor $a$ depends on actor $b$ & CQ9 & Does the actor depend on any actors?& INTRO\\
\hline
AS10 & Task $T_1$ decomposes into tasks $T_2,\ldots,T_n$ & CQ10a & Does the task decompose into other tasks?& REPLACE\\
 &  & CQ10b & Is the decomposition type correct? (AND/OR/XOR)& REPLACE\\
\hline
AS11 & Task $T$ contributes negatively to softgoal $S$& CQ11 & Does the task contribute negatively to the softgoal?& DISABLE\\
\hline
\hline
AS12 & Element $IE$ is relevant & CQ12 & Is the element relevant/useful? & DISABLE\\
\hline
AS13 & Element $IE$ has name $n$ & CQ13 & Is the name clear/unambiguous? & REPLACE\\
\hline
\hline
- & - & Att & Generic counterargument & ATTACK\\
\hline
\end{tabularx}
\caption{List of argument schemes (AS0-AS13, left column), critical questions (CQ0-CQ12, middle column), and the effect of answering them (right column).}
\label{table:argument-schemes}
\end{table*}


\subsection{Metamodel}
\label{sect:metamodel}

Figure~\ref{fig:metamodel} depicts the RationalGRL metamodel linking the main elements of our argumentation extension to the main GRL elements. We describe the metamodel bottom-up, starting with the GRL package.\todo{F}{all}{Fig. 6 is now far removed from where it is mentioned in the text, we should fix this when paper is nearly finished}

The GRL package of our metamodel consists of the core GRL concepts, which constitute a part of
the URN metamodel from Recommendation Z.151~\cite{URN}. These concepts represent the abstract grammar of the language, independently of the notation. This metamodel also formalizes the GRL concepts and constructs introduced earlier\footnote{Note that for readability, some GRL concepts ave been omitted from the figure. For instance, a GRL contribution can have a qualitative strength, ranging from ``Make'' to ``Break''. Since these concepts are not relevant to our framework, they have been omitted.}.

The GRL specification consists of \textsf{GRLModelElements}, which can be either \textsf{GRLLinkableElements} or \textsf{ElementLinks}. A \textsf{GRLLinkableElement} can again be specialized into an \textsf{Actor} or an \textsf{IntentionalElement} (which is either a \textsf{Softgoal}, \textsf{Goal}, \textsf{Task}, \textsf{Resource}, or a \textsf{Belief}). Intentional elements can be part of an actor, and \textsf{GRLLinkableElements} are connected through \textsf{ElementLinks} of different types (i.e., \textsf{Contribution, Decomposition}, or \textsf{Dependency}). Note that actors can be connected through links as well, which is done with \textsf{Dependency} links. 

The Argumentation package depicts the concepts we introduced in the previous sections. An \textsf{ArgumentScheme} represents an (uninstantiated) scheme containing variables that can be replaced with intentional elements. \textsf{CriticalQuestions} are possible ways to attack or elaborate an argument scheme. As such, each critical question applies to exactly one scheme, but each scheme can be elaborated or attacked through multiple critical questions. When an argument scheme is instantiated, we obtain an \textsf{Argument}. Therefore, each argument is associated with exactly one scheme, but a scheme can be instantiated in multiple ways. When a critical question is answered, we obtain an \textsf{AttackLink}. Each \textsf{AttackLink} is associated with at most one critical question, but a critical question can be used to attack multiple arguments.  Note that an \textsf{AttackLink} can also be associated with no critical questions. This allows the user to create attacks between arguments, which do not necessarily correspond to one of the critical questions. A \textsf{RationalGRLDiagram} is composed out of arguments and attack relations.

There is only one link between the \textsf{GRL} package and the \textsf{Argumentation} package, but it is a very important one. The link specifies that each \textsf{GRLModelElement} is in fact an argument. This means that each model element inherits the \textsf{AcceptStatus} as well, allowing GRL elements to be accepted, rejected, or undecided. This, furthermore, means that argument schemes can be applied to all GRL elements, capturing the intuition that each GRL element can be regarded as an instantiated argument scheme. Note that besides arguments about elements of the GRL model, we also have a \textsf{GenericArgument} which is simply a counter-argument to an existing argument that does not relate to any of the GRL elements, but can come from an external source, for instance a piece of evidence or an expert opinion. We will see various examples of such arguments in the next sections.

\begin{figure*}[h!]
\includegraphics[width=\textwidth]{metamodel/metamodel}
\caption{The RationalGRL metamodel}
\label{fig:metamodel}
\end{figure*}

\subsection{RationalGRL Language}

The RationalGRL language is an extension of GRL. The prototype tool we developed, thus, contains various new elements and a new relationship. The legends of the these additions are shown in Figure~\ref{fig:rationalgrllegend}.

In the following subsections, we first explain the new visual language in more detail

RationalGRL is an extension of GRL, and adds the following new elements (Figure~\ref{fig:rationalgrllegend}):
\begin{itemize}
\item \emph{Argument}: This represents an argument created by answering one of the critical questions which disables a GRL element, or counter-attacks a previous argument. %SG: I thought arguments can also do replace, or intro GRL elements. So why in this description we are very limiting? 
\item \emph{Disabled GRL element}: If a GRL element is attacked by an argument, which itself is not attacked, then this GRL element will be disabled, meaning that it does not play a role in the analysis of the GRL model. 
\item \emph{Refined GRL Element}: Not all critical questions attack a GRL element. It is also possible that a critical question \emph{replaces} an existing element (for instance, by clarifying the name of the element), or that it leads to the \emph{introduction} of a new element. In these cases, the corresponding GRL element is shown with a striped background. If the user clicks on this element, a details pane shows up containing the history of refinements (see Section~\ref{sect:tool}). %SG: Can we have a separate element for the intro? or at least explain it in this text more explicitly?
\item \emph{Attack Link}: An attack link can occur between an argument and a GRL element, or between two elements. It means that the source argument attacks the target GRL element or argument.
\end{itemize} %SG: what about the links? The links can also get attacked right? or have some arguments? what do we do with those?

\begin{figure}[h]
\centering
\includegraphics[width=0.35\textwidth]{img/legend}
\caption{The new elements and link of RationalGRL}
\label{fig:rationalgrllegend}
\end{figure}

We manually created RationalGRL models from argument schemes and critical questions found in each transcript, to verify whether the arguments put forward by the participants were sufficiently informative. An example of such a model is shown in Figure~\ref{fig:transcripts:grl}. This figure shows a simplified version of the actual model in order to improve the presentation, but the full models can be found back in our repository. 

\begin{figure*}[h!]
\includegraphics[width=\textwidth]{img/Fig6}
\caption{The GRL model manually constructed from transcript $t_1$. Green dots indicate accepted underlying arguments, red dots indicate rejected underlying arguments. Elements and relationships with no dot have been inferred by us.} %TODO We need to update the caption
\label{fig:transcripts:grl}
\end{figure*} 