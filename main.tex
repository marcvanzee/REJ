\documentclass[11.5pt,two column]{llncs}

\setcounter{tocdepth}{3}
\usepackage{algorithm}
\usepackage{algpseudocode}
\usepackage{amssymb, amsmath}
\usepackage{array,url}
\usepackage{authblk}
\usepackage{courier}
\usepackage{enumitem}
\usepackage{graphicx}
\usepackage{helvet}
\usepackage{multicol}
\usepackage{multirow}
\usepackage{sectsty}
\usepackage{tikz}\usetikzlibrary{arrows}
\usepackage{times}
\usepackage{url}
\usepackage{xcolor}

% change the appearance of the tikz arrow for the argumentation networks
% and some other settings to make all the graphs look similar
\tikzset{>=latex,every node/.style={circle, minimum size=0.5cm, draw=black}} 

\addtolength{\oddsidemargin}{-0.85in}
\addtolength{\evensidemargin}{-0.85in}
\addtolength{\textwidth}{1.7in}
\addtolength{\topmargin}{-.7in}
\addtolength{\textheight}{1.3in}
\setlength{\columnsep}{0.3in}
\newcommand{\Mark}[1]{\textsuperscript{#1}}

% TODO command
\newcommand\todo[4][]{%
	\ifthenelse{\equal{#1}{resolved}}{%
		% nothing
	}{%
		{\bf\color{red}TODO for #3\textcolor{gray}{(by #2)}: #4}%
	}%
}


% TO TURN OFF TODOS UNCOMMENT THE FOLLOWING:
% \renewcommand\todo[4][]{}

% use like this:
% \todo{floris}{marc}{Can you please update figure 2?} - Floris is asking Marc to update figure 2, this will show in red in the PDF
% \todo[resolved]{floris}{marc}{done} - Marc has updated fig 2 (or gives a reason why he doesn't want to uppdate the figure. This wiull show only in the tex


\renewcommand\Authfont{\fontsize{10}{14.4}\selectfont}
\renewcommand\Affilfont{\fontsize{9}{10.8}\selectfont}

\sectionfont{\fontsize{12}{15}\selectfont}
\subsectionfont{\fontsize{10}{15}\selectfont}

\begin{document}
\twocolumn[{%
 \centering
 \LARGE RationalGRL: A Framework for Argumentation and Goal Modeling \\[1.5em]
 \large Marc van Zee\Mark{1},
        Floris Bex\Mark{2},
        and Sepideh Ghanavati\Mark{3}\\[1em]
 \normalsize
 \begin{tabular}{*{3}{>{\centering}p{.33\textwidth}}}
  \Mark{1}Computer Science and Communication (CSC) & \Mark{2}Department of Information and Computing Sciences & \Mark{3}Department of Computer Science \tabularnewline
  University of Luxembourg & Utrecht University & Texas Tech University\tabularnewline
  \url{marcvanzee@gmail.com} & \url{f.j.bex@uu.nl} & \url{sepideh.ghanavati@ttu.edu}
 \end{tabular}\\[3em] % some more space after the title part
}]

\begin{abstract}
Goal modeling languages capture the relations between an information system and its environment using high-level goals and their relationships with lower level goals and tasks. The process of constructing a goal model usually involves discussions between a requirements engineer and a group of stakeholders. While it is possible to capture part of this discussion process in the goal model, for instance by specifying alternative solutions for a goal, not all of the arguments can be found back in the resulting model. For instance, reasons for accepting or rejecting an element or a relation between two elements are not captured. In this paper, we investigate to what extent argumentation techniques from artificial intelligence can be applied to goal modeling. We apply the argument scheme for practical reasoning (PRAS), which is used in AI to reason about goals to the Goal-oriented Requirements Language (GRL). We develop a formal metamodel for the new language, link it to the GRL metamodel, and we implement our extension into jUCMNav, the Eclipse-based open source tool for GRL.
\end{abstract}

\newpage

\paragraph{Keywords} Goal modeling $\cdot$ Argumentation $\cdot$ Practical Reasoning $\cdot$ Goal-oriented requirements engineering

\section{Introduction}
\label{sect:introduction}

Requirements Engineering (RE) is an approach to assess the role of a future information system within a human or automated environment. An important goal in RE is to produce a consistent and comprehensive set of requirements covering different aspects of the system, such as general functional requirements, operational environment constraints, and so-called non-functional requirements such as security and performance. 

Among the initial activities in RE are the ``early-phase'' requirements engineering activities, which include those that consider how the intended system should meet organizational goals, why it is needed, what alternatives may exist, what the implications of the alternatives are for different stakeholders, and how the interests and concerns of stakeholders might be addressed~\cite{yu1997towards}. These activities fall under the umbrella of goal modeling. There are a large number of established RE methods using goal models in the early stage of requirements analysis (e.g.,~\cite{liu2004designing,donzelli2004goal,dardenne1993goal,chung2012non,castro2002towards}, overviews can be found in~\cite{van2001goal,kavakliL05}). Several goal modeling languages have been developed in the last two decades as well. The most popular ones include $i*$~\cite{Yu:1997:TMR:827255.827807}, Keep All Objects Satisfied (KAOS)~\cite{van2008requirements}, the NFR framework~\cite{chung2012non}, \textsc{Tropos}~\cite{giorgini2005goal}, the Business Intelligence Model (BIM)~\cite{horkoff2014strategic}, and the Goal-oriented Requirements Language (GRL)~\cite{Amyot:2010:EGM:1841349.1841356}.

A goal model is often the result of a discussion process between a group of stakeholders. For small-sized systems, goal models are usually constructed in a short amount of time, involving stakeholders with a similar background. Therefore, it is often not necessary to record all of the details of the discussion process that led to the final goal model. However, goal models for many complex, real-world information systems -- e.g., air-traffic management systems, systems that support industrial production processes, or government and healthcare services -- are not constructed in a short amount of time, but rather over the course of several workshops with stakeholders and requirements engineers. In such situations, failing to record the discussions underlying a goal model in a structured manner may harm the success of the RE phase of system development. 

%First, it is well-known that stakeholders' preferences are rarely absolute, relevant, stable, or consistent~\cite{march1978bounded}. Therefore, it is possible that a stakeholder changes his or her opinion about a modeling decision in between two goal modeling sessions, which may require revisions of the goal model. If previous preferences and opinions are not stored explicitly, it is not straightforward to remind stakeholders of their previous opinions which can result in unnecessary discussions and revisions. As the number of participants increases, revising the goal model based on changing preferences can take up a significant amount of time. 

%Second, other stakeholders, such as new developers on the team who were not the original authors of the goal model, may have to make sense of the goal model, for instance, to use it as an input in a later RE stage or at the design and development phase. If these users have no knowledge of the underlying rationale of the goal model, it may not only be more difficult to understand the model, but they may also end up having the same discussions as the previous group of stakeholders.

%Third, alternative and different ideas and opposing views that could potentially have led to different goal models could be lost. For instance, a group of stakeholders specifying a goal model for a user interface may decide to reduce the two goals ``easy to use'' and ``fast'' to one goal ``easy to use''. Thus, the resulting goal model merely contains the goal ``easy to use'', but the discussion as well as the decision to reject the goal ``fast'' are lost. This leads to a poor understanding of the problem and solution domain. In fact, empirical data suggest that this is an important reason of RE project failure~\cite{curtis1988field}. 

%Fourth and finally, in goal models in general the rationale behind any modeling decision is static and changing these rationales does not immediately impact the goal models. That is, current goal modeling languages have limited support for reasoning about changing beliefs and opinions, and their effect on the goal model. A stakeholder may change his or her opinion, but it is not always directly clear what its effect is on the goal model. Similarly, with existing goal modelling languages one can change a part of the goal model, but it is not possible to reason about whether or not this new goal model is consistent with the underlying beliefs and arguments. This becomes more problematic if the participants constructing the goal model change, since modeling decisions made by one group of stakeholders may conflict with the underlying beliefs put forward by another group of stakeholders.

%FB: I changed the text above to that below. The above 4 reasons overlap in some cases, I tried to make this more concise.

The first difficulty is that the goal modeling phase, particularly in large projects, is dynamic and that goal models continuously change and evolve. Stakeholders' preferences are rarely absolute, relevant, stable, or consistent~\cite{march1978bounded}, and stakeholders may change their opinion about a modeling decision in between two goal modeling sessions, which in turn may require revisions of the goal model. If the rationales behind these revisions are not properly documented, alternative ideas and opposing views that could potentially have led to different goal models are lost, as the resulting goal model only shows the end product of a long process and not the discussions during the modeling process. Furthermore, other stakeholders, such as developers who were not the original authors of the goal model, may have to make sense of a goal model in order to, for example, use it as input in a later RE stage or in the development phase. If preferences, opinions and rationales behind the goal models are not stored explicitly, it may not only be more difficult to understand the model, but the stakeholders may also end up having the same unnecessary discussions throughout the goal modeling phase.

A further problem is that the rationale behind goal modeling decisions is usually static, that is, current goal modeling languages have limited support for reasoning about changing beliefs and opinions, and their effect on the goal model. A stakeholder may change his or her opinion, but it is not always directly clear what its effect is on the goal model. Similarly, with existing goal modelling languages one can change a part of the goal model, but it is not possible to reason about whether or not this new goal model is consistent with the underlying beliefs and arguments. This becomes even more problematic if the stakeholders constructing the goal model change, since modeling decisions made by one group of stakeholders may conflict with the underlying beliefs of another group of stakeholders. The disconnect between the goal models and their underlying beliefs and opinions may further lead to a poor understanding of the problem and solution domain, which is an important reason of RE project failure~\cite{curtis1988field}. 

To summarize, what is needed is a way of recording the rationales (beliefs, opinions, discussions, ideas) underlying a goal model. It should be possible to see how these rationales changed during the goal modeling process, and the rationales should be clearly linked to the various elements of the resulting goal model. In order to be able to do this, we propose a framework with tool-support that combines traditional goal modeling approaches with argumentation techniques from Artificial Intelligence (AI) research~\cite{BenchCaponDunne2007}. We have identified \textbf{five important requirements} for our framework: 

\begin{enumerate}
\item 
The argumentation techniques should be close to the actual discussions of stakeholders or designers in the early requirements engineering phase.
\item 
The framework must have formal traceability links between elements of the goal model and underlying arguments.
\item 
Using these traceability links, it must be possible to compute the effect of changes in the underlying argumentation on the goal model, and vice versa.
\item 
There should be a methodology for the framework to guide the practitioners in its application in real cases.
\item 
The framework must have software tool support.
\end{enumerate}

%In this context, the main research question is: 

%\begin{quote}
%\textbf{RQ.} What are the constructs, mechanisms and rules for developing a framework that formally captures the discussions between stakeholders such that it can generate goal models?
%\end{quote}

%FB: I deleted the research question. I think that the requirements already nicely capture exactly what our goal is in the paper. We can rephrase this goal as an RQ, but what does this buy us? Also, the RQ introduces new terminology (construct, mechanisms, rules), to which we have to explicitly refer later in the paper (e.g. we found mechanisms A, B, rules R1, R2, etc.). 

Following on from our previous work \cite{vanzee-etal:renext2015,vanZee-etal:er2016}, we develop a framework called \emph{RationalGRL}, which combines the Goal-oriented Requirements Language (GRL)~\cite{Amyot:2010:EGM:1841349.1841356} with a technique from argumentation theory and discourse modeling called \emph{argument schemes} (or argumentation schemes~\cite{walton-etal2008}). Argument schemes are reusable patterns of reasoning that capture the typical ways in which humans argue and reason. Associated with argumentation schemes are so-called \emph{critical questions}, which can point to typical sources of doubt or implicit assumptions people make when arguing in a particular way. Argument schemes they are very well suited for modeling discussions about a goal model, as they can guide users in systematically deriving conclusions and making assumptions explicit~\cite{murukannaiah2015}. 

One argument scheme that is important when reasoning about goals is the argument scheme for practical reasoning~\cite{walton1990,atkinson2007}, which has been used for, among other things, dialogues about safety critical actions \cite{tolchinsky2012deliberation} and software design discussions~\cite{BlackEtal2013}. Inspired by the work on practical reasoning from Artificial Intelligence, most notably Atkinson and Bench-Capon~\cite{atkinson2007}, we have developed a list of argument schemes that can be used to analyse and guide stakeholders' discussions about goal models. Our approach thus provides a rationalization to the elements of the goal model in terms of underlying arguments, and helps in understanding why parts of the model have been accepted and others have been rejected. Our list of argument schemes was constructed by performing an extensive case study in which we analyzed a set of transcripts containing more than 4 hours of discussions among designers of a traffic simulator information system. This ensures that the argumentation schemes we propose are close to actual real-world discussions stakeholders have (\textbf{requirement 1}).  

The meta-model of the RationalGRL framework clearly specifies the traceability links between the arguments based on the schemes and the GRL models (\textbf{requirement 2}). In addition to this meta-model, we provide formal semantics for RationalGRL by formalising the GRL language in propositional logic and rendering arguments about a GRL model as a formal argumentation framework~\cite{Dung1995}. We then formally capture the link between argumentation and goal modelling as a set of algorithms for applying argument schemes and critical questions about goal models. These formal traceability links allow us to compute the effect of the arguments and counterarguments proposed in a discussion on a GRL model (\textbf{requirement 3}). In other words, we can determine whether the elements of a GRL model are acceptable given potentially contradictory opinions of stakeholders. Thus, we add a new formal evaluation technique for goal models that allows us to assess the \emph{acceptability} of elements of a goal model (in addition to their \emph{satisfiability}~\cite{Amyot:2010:EGM:1841349.1841356}).

Because we want RE practitioners in the field to be able to use our RationalGRL framework (\textbf{requirement 4}), we also propose a methodology for using RationalGRL, which consists of developing goal models and posing arguments based on schemes in an integrated way. To show that the RationalGRL methodology can be used in a real case, we illustrate the steps of our methodology with the traffic simulator case study. Finally, we have developed a web-based prototype\footnote{\url{insertURL}} for building goal models and arguing about them, which acts as a supporting tool to the RationalGRL methodology (\textbf{requirement 5}). 

\todo{F}{all}{check text below when we finish the paper}
 
The rest of this article is organized as follows. Section~\ref{sect:background} introduces our running example, the Goal-oriented Requirements Language (GRL)~\cite{Amyot:2010:EGM:1841349.1841356}, the Practical Reasoning Argument Scheme (PRAS)~\cite{atkinson2007} and discusses some of our previous work on combining GRL and PRAS. Section~\ref{sect:overview} provides a brief and high-level overview of our framework, together with a metamodel and the methodology. Section~\ref{sect:gmas} contains an in depth explanation of how we obtained an initial set of argument schemes and critical questions by annotating transcripts from discussions about an information system, and in Section~\ref{sect:examples} we provide several examples of these schemes and questions. In Section~\ref{sect:formalframework} we provide formal semantics for GRL and show how argumentation semantics \cite{Dung1995} can be used to compute which arguments are accepted and which are rejected. We also develop various algorithms for the argument schemes in this section. In Section~\ref{sect:tool} we provide a brief overview of the prototype tool we developed for RationalGRL. Finally, Section~\ref{sect:discussion} contains a discussion, covering related work, future work, and a conclusion.
\section{Background: Goal-oriented Requirements Language and Argumentation}
\label{sect:background}

In this section, we first introduce our running example, after which we introduce the Goal-oriented Requirements Language (GRL)~\cite{Amyot:2010:EGM:1841349.1841356}, which is the goal modeling language we use to integrate with the argumentation framework. Next, we introduce argumentation: we discuss the \emph{practical reasoning argument scheme (PRAS)}~\cite{atkinson2007}, an argument scheme that is used to form arguments and counter-arguments about situations involving goals, and we give informal examples of how argument and counterargument can influence the status of beliefs about goals. Finally, we briefly discuss the possibilities integrating PRAS and GRL.  %This will be our starting point in the next section.

\subsection{Running example: Traffic Simulator}
\label{sect:goals:runningexample}

We use a traffic simulator design case to explain the concepts and framework in this paper. Our examples and case study are based on a recent series of experiments by Schriek et al. \cite{SchriekEtal2016}, who in turn base their work on the so-called Irvine experiment~\cite{UCIworkshop}, which presents a well-known design reasoning assignment in software engineering. In this assignment (see Appendix~\ref{sect:designprompt}), designers are provided with a problem description, requirements, and a description of the desired outcomes: The client of the project is Professor E, who teaches civil engineering courses at an American university. In order for the professor to teach students the various theories concerning traffic (such as queuing theory), traffic simulator software needs to be developed in which students can create visual maps of an area, regulate traffic, and so forth. Schriek and colleagues asked designers (groups of students) to discuss the requirements of this traffic simulator. These discussions were recorded and transcribed, and we used these transcripts for an extensive case study (Section \ref{sect:} on the basis of which we developed our RationalGRL framework (Section \ref{sect:gmas}). Furthermore, we also use the traffic simulator case for a simple running example in this section (Figures \ref{fig:example-small}, \ref{fig:pras:example}). 

\subsection{Goal-oriented Requirements Language (GRL)}
\label{sect:background:grl}
GRL is a visual modeling language for specifying intentions, business goals, and non-functional requirements of multiple stakeholders \cite{Amyot:2010:EGM:1841349.1841356}. GRL is part of the User Requirements Notation, an ITU-T standard, that combines goals and non-functional requirements with functional and operational requirements (i.e. use case maps). GRL can be used to specify alternatives that have to be considered, decisions that have been made, and rationales for making decisions. A GRL model is a connected graph of intentional elements that optionally are part of the actors. All the elements and relationships used in GRL are shown in Figure~\ref{fig:grl_legend}.

\begin{figure}[ht]
\centering
\includegraphics[width=\columnwidth]{img/Example1-new.pdf}
\caption{Partial GRL Model of the traffic simulator example}
\label{fig:example-small}
\end{figure} 

\begin{figure*}[ht]
\centering
\includegraphics[scale=0.6]{img/grl_legend}
\caption{Basic elements and relationships of GRL}
\label{fig:grl_legend}
\end{figure*}

Figure~\ref{fig:example-small} illustrates a simplified GRL diagram from the traffic simulator design exercise. An actor (\includegraphics[scale=1]{img/actor}) represents a stakeholder of a system or the system itself (\emph{Traffic Simulator}, Figure~\ref{fig:example-small}). Actors are holders of intentions; they are the active entities in the system or its environment who want goals to be achieved, tasks to be performed, resources to be available, and softgoals to be satisfied. Softgoals (\includegraphics[scale=1]{img/softgoal}) differentiate themselves from goals (\includegraphics[scale=1]{img/goal}) in that there is no clear, objective measure of satisfaction for a softgoal whereas a goal is quantifiable, often in a binary way. Softgoals (e.g. \emph{Realistic simulation}) are often related to non-functional requirements, whereas goals (such as \emph{Generate Cars}) are related to functional requirements. Tasks (\includegraphics[scale=1]{img/task}) represent solutions to (or operationalizations of) goals and softgoals. In Figure~\ref{fig:example-small}, we have the two tasks \emph{Create new cars} and \emph{Keep same cars}: in order to achieve goal \emph{Generate cars}, the simulation can either constantly generate new ones or keep the same cars and have them reappear after they disappear off screen. In order to be achieved or completed, softgoals, goals, and tasks may require resources (\includegraphics[scale=1]{img/resource}) to be available (e.g., \emph{Car Objects}). Finally, design rationale can be captured using beliefs (\includegraphics[scale=0.3]{img/belief}). For example, the idea that the simulation should be \emph{Not scientifically correct} is captured as a belief in the model. 

Different links connect the elements in a GRL model. AND, IOR (Inclusive OR), and XOR (eXclusive OR) decomposition links (\includegraphics[scale=1]{img/decomposition}) allow an element to be decomposed into sub-elements. In Figure~\ref{fig:example-small}, the goal \emph{Generate cars} is XOR-decomposed to the tasks \emph{Create new cars} and \emph{Keep same cars}, as they are alternative ways of achieving the goal \emph{Generate cars}. Contribution links (\includegraphics[scale=1]{img/contribution}) indicate desired impacts of one element on another element. A contribution link has a qualitative contribution type or a quantitative contribution. Task \emph{Create new cars} has a \emph{help} qualitative contribution to the task \emph{Realistic simulation}, and a \emph{hurt} qualitative contribution to the task \emph{Simple design}. Dependency links (\includegraphics[scale=1]{img/dependency}) model relationships between actors or resources. Here, the goal \emph{Generate cars} depends on the resource \emph{Car objects}. 

GRL is based on $i*$~\cite{Yu:1997:TMR:827255.827807} and the NFR Framework~\cite{chung2012non}, but it is not as restrictive as $i*$. Intentional elements and links can be more freely combined, the notion of agents is replaced with the more general notion of actors, i.e., stakeholders, and a task does not necessarily have to be an activity performed by an actor, but may also describe properties of a solution. GRL has a well-defined syntax and semantics. Furthermore, GRL provides support for providing a scalable and consistent representation of multiple views/diagrams of the same goal model (see~\cite[Ch.2]{Ghanavati2013} for more details). GRL is also linked to Use Case Maps via URNLinks (\includegraphics[scale=1]{img/urnlink}), which provide traceability between concepts and instances of the goal model and behavioral design models. Multiple views and traceability links are a good fit with our current research: we aim to add traceability links between intentional elements and their underlying arguments. 

GRL has six evaluation algorithms which are semi-automated and allow the analysis of alternatives and design decisions by calculating the satisfaction value of the intentional elements quantitatively, qualitatively or in a hybrid way. jUCMNav, GRL tool-support, also allows for adding new GRL evaluation algorithms~\cite{jUCMNav}. GRL also has the capability to be extended through metadata, links, and external OCL constraints. This allows GRL to be used in many domains without the need to change the whole modeling language. 

The GRL model in Figure~\ref{fig:example-small} shows the softgoals, goals, tasks and the relationship between the different intentional elements in the model. However, the rationales and arguments behind certain intentional elements are not shown in the GRL model. Some of the questions that might be interesting to know about are the following:

\begin{itemize}
	\item Why is belief \emph{Not scientifically correct} not linked to any of the goals or tasks? 
	\item What does \emph{Keep same cars} mean?
	\item Why does task \emph{Create new cars} contribute negatively to \emph{Simple design} and positively to \emph{Realistic simulation}?
	\item Why does \emph{Generate cars} XOR-decompose into two tasks?
\end{itemize}

These are the type of the questions that we cannot answer by just looking at GRL models. The model in Figure~\ref{fig:example-small} does not contain information about discussions that led to the resulting model, such as various clarification steps for the naming, or alternatives that have been considered for the relationships. The idea behind the original GRL specification is that beliefs can be used to capture such design rationales that make later justification and review of a model easier. However, beliefs cannot be connected to links - this makes answering the third and fourth question above impossible. Furthermore, beliefs are after-the-fact design rationales and do not capture the types of critical questions given above. 

\subsection{Practical Reasoning Argument Scheme (PRAS)}
\label{sect:background:pras}

Reasoning about which goals to pursue and actions to take is often referred to as \emph{practical reasoning}, and has been studied extensively in philosophy and artificial intelligence. One approach is to capture practical reasoning with argument schemes~\cite{walton1990}. Applying an argument scheme results in an argument in favor of, for example, taking an action. This argument can then be tested with critical questions about, for instance, whether the action is possible given the situation, and a negative answer to such a question leads to a counterargument to the original argument for the action. 

Atkinson and Bench-Capon~\cite{atkinson2007} develop and formalize the \emph{Practical Reasoning Argument Scheme} (PRAS). A simplified version of this argument scheme is as follows:

\begin{itemize}
\item[] $G$ is a goal,
\item[] Performing action $A$ realizes goal $G$,
%\item[] Which will contribute positively to the softgoal $S$
\item[] \textit{Therefore} 
\item[] Action $A$ should be performed
\end{itemize}

Here, $G$ and $A$ are variables, which can be instantiated with concrete goals and actions to provide a specific practical argument. For example, a concrete argument about the traffic simulator is as follows: 
\begin{itemize}
\item[] \emph{Generate Cars} is a goal,
\item[] Performing action \emph{Keep same cars} realizes goal \emph{Generate cars}, 
\item[] \underline{Therefore} 
\item[] Action \emph{Keep same cars} should be performed
\end{itemize}

Note that PRAS is an argument scheme that captures a full inference step: ``$G$, $A$ realizes $G$ \emph{Therefore} $A$''. There are, however, also schemes that capture simpler reasoning patterns, such as claims of the form ``$A$ does not realize $G$''. We will discuss these schemes below. 

In argumentation, conclusions which are at one point acceptable can later be rejected because of new information. For example, we may argue that, in fact, performing action \emph{Keep same cars} does not realize goal \emph{Generate cars}, thus giving a counterargument to the above instantiation of PRAS. Atkinson et al.~\cite{atkinson2007} define a set of so-called critical questions that point to typical ways in which an argument based on PRAS can be criticized by. Some examples of critical questions are as follows.

\begin{enumerate}
\item[CQ1] Will the action realize the desired goal?
\item[CQ2] Are there alternative ways of realizing the same goal?
\item[CQ3] Does performing the action have a negative side effect?
\end{enumerate}

The idea is that answers to critical questions are counterarguments to the original PRAS argument. These counterarguments also follow a scheme; for example, a negative answer to CQ1 follows the scheme ``Action $A$ will not realize goal $G$'', which can be instantiated (e.g. ``\emph{Keep same cars} does not realize \emph{Generate cars''}) to form a counterargument to the original argument. 

Another way to criticize an argument for an action is to suggest an alternative action that realizes the same goal (CQ2). For example, we can argue that performing \emph{Create new cars} also realizes the goal \emph{Generate cars}. Also, it is possible that performing an action has a negative side effect (CQ3). For example, while the action \emph{Create new cars} realizes the goal \emph{Generate cars}, it has a negative side effect, namely hurting \emph{Simple design}: having the simulation constantly create new cars is fairly complex design choice. 

In argumentation, counterarguments are said to \emph{attack} the original arguments. Given a set of arguments and attacks between these arguments, we can compute which arguments are accepted and which are rejected using different argumentation semantics~\cite{Dung1995}\footnote{Formal definitions of argumentation frameworks and semantics will be given in section \ref{sect:gmas}. In this section, we will briefly discuss the intuitions behind these concepts.}. Figure \ref{fig:pras:example} shows three arguments from the traffic simulation example, where arguments are rendered as boxes and attack relations as arrows. There are two arguments based on PRAS: argument A1 for \emph{Keep Same Cars} and argument A2 for \emph{Create new cars}. Argument A2 proposes an alternative way of realizing the same goal \emph{Generate cars} with respect to argument A1 and vice versa (cf. CQ2), so A1 and A2 mutually attack each other, denoted by the arrows between A1 and A2. Argument A3 says that \emph{Create new cars} has a negative effect on \emph{Static Simulation}, so A3 attacks A2, as it points to a negative side-effect of \emph{Create new cars} (CQ3). The intuition here is that an argument is acceptable if any argument that attacks it is itself rejected. In Figure~\ref{fig:pras:example}, argument A3 is accepted because it has no attackers. This makes A2 rejected (indicated by the lighter grey color), because its attacker A3 is accepted. A1 is then also accepted, since its only attacker, A2, is rejected. 

\begin{figure}[h]
\centering
\includegraphics[]{img/fig_AF1.pdf}
\caption{PRAS arguments and attacks in the traffic simulation example.}
\label{fig:pras:example}
\end{figure}


\iffalse
\begin{figure}[ht!]
\centering
\begin{tikzpicture}
        \node[minimum size=1cm] (att3) [argNodeIN] at (-0.5,-3.4) {\argtext{A1}{Action \emph{Keep Same Cars} realizes goal \emph{Generate Cars}\\ \emph{\underline{Therefore}} Action \emph{Keep Same Cars} should be performed}};
        \node[minimum size=1cm] (att1) [argNodeOUT] at (-0.5,0) {\argtext{A2}{Action \emph{Create New Cars} realizes goal \emph{Generate Cars}\\ \emph{\underline{Therefore}} Action  \emph{Create New Cars} should be performed}};
        \node[minimum size=1cm] (att2) [argNodeIN] at (3.5,0) {\argtext{A3}{\emph{Create New Cars} has a negative effect on \emph{Static Simulation}}};
         \path
    (att2) edge [attackLink] (att1)
    (att1) edge [attackLink,<->] (att3);
\end{tikzpicture}
\caption{Arguments and attacks in the traffic simulation example.}
\label{fig:pras:example}
\end{figure}
\fi


Looking at PRAS and its critical questions, one can see how it could be used to argue about goals and actions or, more specifically, about goal models. However, we cannot literally use PRAS and its critical questions, as there are elements of the GRL language, such as actors and resources, which cannot be found in PRAS. Furthermore, it is not directly clear whether the critical questions as proposed by Atkinson and Bench-Capon~\cite{atkinson2007} actually apply to GRL models. In fact, our case study (Section \ref{sect:gmas}) shows that when discussing requirements, people very often do not structure their reasoning nicely in the way that PRAS presents it. That is, you do not see the discussants setting up an argument ``We have goal $G$, $A$ realizes $G$ \emph{Therefore} we should perform $A$''. A typical discussion is much more unstructured, as is clear from the transcript excerpts in Appendix~\ref{sect:transcripts:excerpts}. So if we would use the version of PRAS presented in this section for our argumentation, we would violate requirement 1: The argumentation techniques should be close to the actual discussions of stakeholders or designers in the early requirements engineering phase. Our solution was to develop our own set of argument schemes and critical questions by analyzing transcripts of discussions about the traffic simulator. This set of schemes and questions and our case study are described in the next section. 


\section{Preview of the Framework}
\label{sect:preview}

\todo{Marc}{Marc}{Give an example showing how our framework works without the technical details.}
\section{Argument Schemes for Goal Modeling (GMAS): First Iteration}
\label{sect:gmas}

In this section we develop an initial set of argument schemes and critical questions for goal modeling, based by the argument scheme for practical reasoning as described in the previous section. In the next section, we validate and improve this list by annotating arguments found in transcripts of discussions about an information system.

\begin{table*}[h]
\centering
\begin{tabular}{|l|l|l|l|l|}
\hline
\multicolumn{2}{|c|}{\textbf{Argument scheme}} & \multicolumn{2}{c|}{\textbf{Critical Questions}}\\
\hline
AS0 & Actor $a$ is relevant & CQ0 &Is the actor relevant?\\
\hline
AS1 & Actor $a$ has resource $R$ & CQ1 &Is the resource available?\\
\hline
AS2 & Actor $a$ can perform task $T$ & CQ2 &Is the task possible?\\
\hline
AS3 & Actor $a$ has goal $G$ & CQ3 & Can the desired goal be realized?\\
\hline
AS4 & Actor $a$ has softgoal $S$ & CQ4 & Is the softgoal a legitimate softgoal?\\
\hline
\hline
AS5 & Task $T$ contributes to goal $G$ & CQ5a & Does the task contribute to the desired goal?\\
& & CQ5b & Are there alternative ways of realizing the same goal?\\
\hline
AS6 & Task $T$ contributes to softgoal $S$& CQ6a & Does the task contribute to the softgoal?\\
&& CQ6b & Are there alternative ways of contributing to the same softgoal? \\
&& CQ6c & Does the task have a side effect which contribute negatively to some other softgoal?\\
&& CQ6d & Does the task contribute to some other softgoal?\\
\hline
AS7 & Goal $G$ contributes to softgoal $S$ & CQ7a & Does the goal contribute to the softgoal?\\
&& CQ7b & Does the goal contribute to some other softgoal?\\
\hline
AS8 & Resource $R$ contributes to task $T$ & CQ8 & Is the resource required in order to perform the task?\\
\hline
AS9 & Actor $a$ depends on actor $b$ & CQ9 & Does the actor depend on any actors?\\
\hline
\end{tabular}
\caption{Initial list of argument schemes and critical questions for GRL elements (AS0-AS4) and relationships (AS5-AS16).}
\label{table:argument-schemes}
\end{table*}

\subsection{Argument schemes and critical questions}

Where PRAS consists of the single argument scheme PAS, our approach is to split this into a family of argument schemes, such that each separate argument scheme can be applied to a specific part of a goal model. Similar to PRAS, someone who does not accept the presumptive argument may challenge it by applying critical questions. We have modified Atkinson et al.'s original 16 questions associated with the scheme \cite{atkinson2006argumentation} by removing those questioning elements not part of GRL (\emph{circumstances}), adding questions for additional elements GRL (\emph{resources}), and renaming concepts and relationships (e.g., the \emph{promotes} relationship).

The initial list We developed a total of 10 argument schemes and 18 critical questions, which are shown in table~\ref{table:argument-schemes}. Argument schemes AS0-AS4 can be instantiated for individual GRL elements, and each have a single critical question. CQ2-CQ4 are critical questions by Atkinson rephrased using the GRL terminology, while CQ1 is added in order to question a resource element, which does not appear in PRAS. AS5-AS9 are argument schemes for various GRL relationships between elements. As we mentioned before, this list is not meant to be exhaustive, since GRL does not put restrictions on any of relationships, so in theory any type of element can be combined with any type of elements, using any relationship. Here we merely use the critical questions developed by Atkinson et al.
 
Answering a critical question results in the creation of a new argument, which may or may not attack the original arguments, depending on which question is answered. For instance, answering CQ1 with ``no'' results in a new argument attacking the argument that actor $a$ has resource $R$ available. In contrary, answering CQ5b with ``yes'' does not result in an attack on the original argument, but creates a new argument stating that some other task realizes the goal as well. We will return to this issue in more detail in the next section. In the current section, we restrict our analysis to the development of appropriate argument schemes and critical questions.

\section{GMAS: Second iteration}
\label{sect:gmas:2}

The argument schemes and critical questions of Table~\ref{table:argument-schemes} are based on the theoretical work by Atkinson, and our own observations about the differences with GRL. However, it is unclear whether this list is exhaustive, or even whether the schemes and questions are actually used in practice. In order to evaluate this, we annotate transcripts of discussions between students with the arguments occurring in them, and analyze the results in order to refine our list.

\subsection{Empirical evaluation with transcripts}

The transcripts we used are created as part of two master theses on improving design reasoning~\cite{masterthesis1,masterthesis2}.

\paragraph{Subjects} The subjects for the case study are three teams of Master students from the University of Utrecht, following a Software Architecture course. Two teams consist of three students, and one team consists of two students.

\paragraph{Experimental Setup} The assignment used for the experiments is to design a traffic simulator. Designers are provided a problem description, requirements, and a description of the desired outcomes. The original version of the problem descrption~\cite{UCIworkshop} is well known in the field of design reasoning since it has been used in a workshop\footnote{\url{http://www.ics.uci.edu/design-workshop/}}, and transcripts of this workshop have been analyzed in detail~\cite{Petre:2013:SDA:2535028}. Although the concepts of traffic lights, lanes, and intersections are common and appear to be simple, building a traffic simulator to represent these relationships and events in real time turns out to be challenging. Participants were asked to use a think-aloud method during the design session. For the student groups the assignment was slightly adjusted to include several viewpoints as end products in order to conform to the course material~\cite{Bass:2012:SAP:2392670}. The final problem descriptions can be found in Appendix A of Schriek's master thesis~\cite{masterthesis1}. All groups were instructed to apply the functional architecture method (FAM), focusing on developing the Context, the Functional, and the Informational viewpoint of the traffic simulator software. The students had two hours for the tasks, and the transcripts document the entire discussion. The details of the transcripts are shown in table~\ref{table:transcripts:info}.

\begin{table}[ht]
\centering
\begin{tabular}{|l|l|l|l|}
\hline
& transcript $t_1$ & transcript $t_2$ & transcript $t_3$\\
\hline
participants & 2 & 3 & 3\\
\hline
duration & 1h34m52s & 1h13m39s & 1h17m20s\\
\hline
\end{tabular}
\caption{Number of participants and duration of the transcripts.}
\label{table:transcripts:info}
\end{table}

\paragraph{Annotation method} We annotated transcripts with the arguments and critical questions of table~\ref{table:argument-schemes}. If we found arguments or critical questions that did not appear in the list, we added them and counted them as well. Most of the occurrences were not literally found back, but had to be inferred from the context. For instance, if a participant questions an argument with a statement such as ``I don't know about that'', then we interpret this as a critical question.

It is generally known in the argumentation literature that it can be very difficult to annotate arguments correctly.\todo{Marc}{Floris}{add citation} Arguments are often imprecise, lack conclusion, and may be supported by non verbal communication that is not captured in the transcripts. Still, since research on argument extraction in the requirement engineering domains is in its infancy, we believe that our evaluation is useful by itself. Furthermore, our annotation is openly available\footnote{\todo{Marc}{Marc}{provide url}}, we provide parts of our annotation in Appendix~\ref{sect:transcripts:excerpts}, and most of the examples from this article come from the transcripts. In this way, we aim to make our annotation process as transparent as possible.

\subsubsection{Results}

Some examples of annotation can be found in appendix~\ref{sect:transcripts:excerpts}. We found a total of 120 instantiations of the existing argument schemes AS0-AS9 in the transcripts. The most used argument scheme was AS2: ``Actor $A$ has task $T$'', but each argument scheme has been found back in the transcripts at least twice (table~\ref{table:transcripts:results:argumentschemes}). Examples of argument schemes are AS1, an argument for a resource (table~\ref{table:transcript:as1-as8}); AS2, an argument for a task (tables~\ref{table:transcript:as2-cq_star_1-cq2},~\ref{table:transcript:as2-cq_star_2}, and~\ref{table:transcript:as2-as10}); AS4, an argument for a softgoal, AS5, an argument for a goal, AS7, an argument for a contribution from goal to softgoal (table~\ref{table:transcript:as4-as5-as7}); and AS8, an argument for a contribution from resource to task (table~\ref{table:transcript:as1-as8}). 

Of our critical questions, we annotated 9 instantiations. Example of critical questions are CQ0, questioning the relevance of an actor (table~\ref{table:transcript:as0-cq0}) and CQ2, questioning the possibility of a task (table~\ref{table:transcript:as2-cq_star_1-cq2}).

\begin{table}[ht]
\centering
\begin{tabular}{|l|l|l|l|>{\bfseries}l|}
\hline
\textbf{Argument Schemes} & $t_1$ & $t_2$ & $t_3$ & \textbf{total}\\
\hline 
AS0: Actor & 2 & 2 & 5 & 9\\
\hline
AS1: Resource & 2 & 4 & 5 & 11\\
\hline
AS2: Task/action & 20 & 21 & 17 & 58\\
\hline
AS3: Goal & 0 & 2 & 2 & 4\\
\hline
AS4: Softgoal & 3 & 4 & 2 & 9\\
\hline
AS5: Goal decomposes into Task & 4 &0& 4 & 8\\
\hline
AS6: Task contributes to softgoal & 6 & 2 &0& 8\\
\hline
AS7: Goal contributes to softgoal &0& 1 & 1 & 2\\
\hline
AS8: Resource contributes to task & 0 & 4 & 3 & 7\\
\hline
AS9: Actor depends on actor &0& 1 & 3 & 4\\
\hline
\hline
\textbf{TOTAL} & 37& 41 & 42 & 120\\
\hline
\end{tabular}
\caption{Number of occurrences of AS0-AS9 in the transcripts.}
\label{table:transcripts:results:argumentschemes}

\begin{tabular}{|l|l|l|l|>{\bfseries}l|}
\hline
\textbf{Critical questions} & $t_1$ & $t_2$ & $t_3$ & \textbf{total}\\
\hline 			
CQ2: Task is possible? & 2 & 2 & 1 & 5\\
\hline		
CQ5a: Does the task contribute to the the goal? & 0 & 1 & 0 & 1\\
\hline
CQ5b: Alternative ways to realize the same goal? & 1 & 0 & 0 & 1\\
\hline
CQ6b: Task has negative side effects? & 2 & 0 & 0 & 2\\
\hline
\hline
\textbf{TOTAL} & 5 & 2 & 1 & 9\\
\hline
\end{tabular}
\caption{Number of occurrences of critical questions CQ0-CQ9 in the transcripts. Critical questions not appearing in this table were not found back in the transcripts.}
\label{table:transcripts:results:criticalquestions}
\end{table}

Additionally, we identified 85 statements that did not fit our existing argument schemes and critical questions, leading to 2 new argument schemes (39 occurrences) and 8 new critical questions (29 occurrences). The new argument schemes and critical questions are shown in table~\ref{table:transcripts:results:new}. 

\subsubsection{Analysis}

The analysis of the results of our empirical evaluation consists of three parts. First, we analyze the argument schemes, then the critical questions, and finally we analyze the effect of posing a new argument.

\paragraph{Analysis of the argument schemes}
The difference between the initial list of argument schemes and those found back in the transcripts is quite small. We found it surprising that we were able to find back all the schemes in the transcript at least twice, even more since the topic of discussion wasn't goal models, but more generally the architecture of an information system. This gives us an indication that these argument schemes are able to capture arguments used in those type of discussions to some extent. However, we also found the following additional argument schemes:
\begin{itemize}
\item
\emph{AS10: Task decomposes into task.} While our initial list of argument schemes contains a scheme to decompose goals into tasks (AS5), it does not contain one for task decomposition. We found that are large part of the discussions were focused around the tasks that the information system should provide. In the first phase of the discussion, participants often listed a number of tasks the information system should provide (table~\ref{table:transcript:as2-cq_star_1-cq2}), which were then later refined through task decomposition (table~\ref{table:transcript:as2-as10}).
\item 
\emph{AS11: Task contributes negatively to softgoal.}  We found that participants occasionally discuss negative contributions as well (table~\ref{table:transcript:as11}).
\end{itemize}
More generally, we observed that our initial list is rather limited, which is a consequence of the fact that it is derived from PRAS. Since PRAS only considers very specific types of relationships, we are not able to capture many other relationships existing in GRL. GRL has four types of intentional elements (softgoal, goal, task, resource) and four types of relationships (positive contribution, negative contribution\footnote{In fact, a contribution can be any integer in the domain [-100,100], but for the sake of simplicity we only consider two kinds of contributions here.}, dependency, decomposition), allowing theoretically $4^3=64$ different types of argument schemes, of which we currently only consider 5. Our analysis shows that many of these schemes are not often used, but we should at least add the possibility for task decomposition (AS10) and negative contribution (AS11).

\begin{table*}[ht]
\centering
\begin{tabular}{|l|l|l|l|>{\bfseries}l|}
\hline
\textbf{New annotation} & $t_1$ & $t_2$ & $t_3$ & \textbf{total}\\
\hline
AS10: Task decomposes into task & 11 & 14 & 11 & 36\\
\hline
- CQ10a: Does the task decompose into other tasks? & 1 &2 &0&3\\
\hline
- CQ10b: Is the decomposition correct (AND/OR/XOR)? &1 &0& 1 &2\\
\hline
AS11: Task contributes negatively to softgoal&2&1	&0&3\\
\hline
\hline
TI: Topic introduction & 5 & 3 & 7 &15\\
\hline
REM: Removing useless/irrelevant/redundant element & 2 & 3 & 2 &7\\
\hline
CLAR: Clarifying an element &3 &10 & 3 & 16\\
\hline
GEN: Generic counterargument	& 0& 2 & 2 & 4\\
\hline
\textbf{TOTAL}&24&34&25&87\\
\hline
\end{tabular}
\caption{Number of occurrences of new argument schemes and critical questions in the transcripts.}
\label{table:transcripts:results:new}
\end{table*}

\paragraph{Analysis of the critical questions} The difference between the initial list of critical questions and those we found back in the transcripts is much larger than for the critical questions. On the one hand, we found back few of the critical questions we initially proposed. However, this does not mean that they weren't implicitly used in the minds of the participants. If a participants makes an argument for a contribution from a task to a softgoal, it may very well be that she was asking herself the question ``Does the task contribute to some other softgoal?''. However, many of these critical questions are not mentioned explicitly. If we assume this explanation is at least partially correct, then this would mean that critical questions may still play a role when formalizing the discussions leading up to a goal model, and it would be limiting to leave them out of our framework. In the context of tool support, we feel that having these critical questions available may stimulate discussions.

On the other hand, we found back relatively many new critical questions that were not on our initial list. We found back two critical questions for the new argument schemes AS10:
\begin{itemize}
\item \emph{CQ11a: Does the task decompose into other tasks?} (table~\ref{table:transcript:cq:task_decomp}).
\item \emph{CQ11b: Is the decomposition correct? (AND/OR/XOR)} The initial list of critical questions does not distinguish between the type of decomposition that may occur. However, we found that this is sometimes discussed by participants (table~\ref{table:transcript:cq11b}).
\end{itemize}
Moreover, we found four types arguments that we could not classify as an instantiation of an argument scheme or a critical question:
\begin{itemize}
\item \emph{Topic introduction.} Before posing arguments, participants often proposed to discuss a certain topic, for instance by stating ``we have to determine who the users of the system are gonna be'' (table~\ref{table:transcript:as-star})..
\item \emph{Removing useless/irrelevant/redundant element} We found this critical question especially in relation with tasks (table~\ref{table:transcript:as2-cq_star_1-cq2}).
\item \emph{Clarifying an element} Given the large body of literature on clarification \todo{Marc}{Marc}{add citations}, it is not surprising that we found this type of argument relatively often. We do not aim to provide a detailed analysis of the different types of clarification techniques in this article, but we merely want to be able to capture an argument clarifying a previous argument in a course grained way. Such an analysis in the context of argumentation and requirement engineering can be found elsewhere~\cite{Jureta:RE2008}. See table~\ref{table:transcript:as2-cq_star_2} for an example of a clarification argument.

\item \emph{CQ: Generic counterargument.} Participants occasionally posed counterarguments against arguments or critical questions as well, which we could not classify further (table~\ref{table:transcript:as0-cq0-cq_counterarg}).
\end{itemize}

\paragraph{Analysis of the effect of a new argument} When a new argument is put forward, this can have varying effects on the previous arguments. For instance, in the excerpt of table~\ref{table:transcript:as0-cq0}, the answer to the critical questions results in an argument attacking the original argument for actor ``Development team'', which ensures that ``Development team'' is no longer considered to be a relevant actor of the system. In table~\ref{table:transcript:as2-cq_star_2}, answering the critical question also results in an attack on the original argument for task ``car influx'', but at the same time also generates a new argument for a more specific task ``control car influx per road''. In general, we found the following four operations that may be associated with the arguments:

\todo{Marc}{Marc/Floris}{Say that an argument is attacked in different ways and we have to distinguish this}
\begin{itemize}
\item \emph{INTRO: Introduce new element/link.} This operation does not attack anything, but generates new elements. Examples are CQ5b, CQ6c, and \emph{Topic introduction}.
\item \emph{DISABLE: Disable element/link.} This operation generates an attack on an element or link but does not replace it with anything. Examples are CQ0-CQ4, CQ5a, and \emph{Clarifying an element}.
\item \emph{REPLACE: Replace element/link.} This operation replaces the description of the intentional element, or it replaces the type of link (e.g., from positive contribution to negative contribution, or from AND-decomposition or OR-decomposition). An example is \emph{Clarifying an element}.
\item \emph{GENERIC: Generic counterargument.} This operation simply directly attacks an argument. It is different from DISABLE in the sense that it only attacks one argument directly, while DISABLE attacks an argument for an element/link, and all possible previous replacements for it. We explain this in more detail in the next section where we introduce our algorithms.
\end{itemize}

\subsection{Final argument schemes and critical questions}

Our final set of argument schemes are AS0-AS9 (table~\ref{table:argument-schemes} and AS10-AS11 (table~\ref{table:transcripts:results:new}, and the final set of critical questions are CQ0-C9 (table~\ref{table:argument-schemes} and CQ10 (table~\ref{table:transcripts:results:new}. Additionally, we add the four arguments TI, REM, CLAR, and GEN (table~\ref{table:transcripts:results:new}.

In the end of the previous section we distinguished four operations (REPLACE, DISABLE, INTRO, and GENERIC) that can be applied when instantiating an argument scheme or answering a critical question. In table~\ref{table:operation-mappings} we provide a mapping form our argument schemes, critical questions, and additional arguments to these four operations. In the next section, we formalize these operations as algorithms.

\begin{table}[ht]
\begin{tabular}{|l|l|}
\hline
\textbf{Argument} & \textbf{Operation}\\
\hline
AS0-AS11, CQ5b,CQ6b-d,CQ7b,TI& INTRO\\
\hline
CQ0-CQ4,CQ5a,CQ6a,CQ7a,CQ8,CQ9,REM & DISABLE\\
\hline
CLAR & REPLACE\\
\hline
GEN & GENERIC\\
\hline
\end{tabular}
\caption{Mapping from the final argument schemes AS0-AS11, final critical questions CQ0-CQ10, and additional arguments TI, REM, CLAR, GEN to the operations.}
\label{table:operation-mappings}
\end{table}
\section{RationalGRL: Logical Framework}
\label{sect:formalframework}

In Section~\ref{sect:overview}, we have shown through a language definition and informal examples from our case study that it is possible to trace elements of the goal model back to their underlying arguments (\textbf{requirement 2}), and that it is possible to determine the effect of changes in the underlying argumentation on the goal model, and vice versa (\textbf{requirement 3}). A formal rendition of this traceability will be presented in this section, in which we present a logical formalization of RationalGRL. This is done for multiple reasons: (i) Most approaches in formal argumentation (cf. Section~\ref{sect:background:pras}) use formal logic, allowing us to employ existing technique directly in order to compute which arguments are accepted and which are rejected, (ii) we can be more precise about how critical questions are answered, (iii) we can show that RationalGRL models can be translated to valid GRL models and visa versa in a precise way, and (iv) the formal approach is a basis for automating the framework in terms of tool support, which we present in the next section.

In Sections \ref{sect:formalframework:grl} and \ref{sect:formalframework:rationalgrl} we formalize a static representation of our framework based on the metamodel (Figure~\ref{fig:metamodel}). We first provide a formal specification of a GRL model in Section~\ref{sect:formalframework:grl}, and we extend this with arguments and attack links in the second subsection, hereby obtaining a formal specification of a RationalGRL model. In Section~\ref{sect:formalframework:translation} we then present algorithms in order to translate a GRL model into a RationalGRL model, and vice versa. Finally, in Section~\ref{sect:algorithms} we turn to the dynamics of our framework, developing algorithms for instantiating argument schemes and answering critical questions and thus formally capturing the \textsf{INTRO}, \textsf{REPLACE} and \textsf{DISABLE} operations of which examples were previously given in Section~\ref{sect:overview:examples}. 

\subsection{Formal Specification of GRL}
\label{sect:formalframework:grl}

We formalize a GRL model based on the metamodel (Figure~\ref{fig:metamodel}), starting with intentional elements and actors.

Throughout this section, we adopt the convention that variables start with a lowercase letter (e.g, $id$, $i$, $j$, $name$, $goal$), and sets and constants start with an uppercase letter (e.g., $Type, AND, Goal$). We start by defining general elements of our language that we use in subsequent definitions.

\begin{definition}[General definitions]
\label{def:set-definitions}
We define the following sets:
\begin{itemize}
\item $IETypes = \{Softgoal, Goal, Task, Resource\}$,
\item $LinkTypes = \{PosContr, NegContr, Dep, Decomp\}$\footnote{Recall from Section~\ref{sect:background} that for contribution links we only distinguish between positive and negative contributions. Extending the formalization model to include all the GRL values is straightforward but has been omitted here for conciseness.},
\item $Types = IETypes \cup LinkTypes\cup\{Actor, ActIE, GenArg\}$,
\item $Names$ is a finite set of strings,
\item $DecompTypes = \{AND,OR,XOR\}$.
\end{itemize}
\end{definition}
Next, we define an intentional element.

\begin{definition}[Intentional Element]
\label{def:ie}
An intentional element is a tuple $ie = (id, type, name, decomptype)$, where:
\begin{itemize}
\item $id\in \mathbb{N}$ is a unique identifier for the element,
\item $type\in IETypes$ specifies the type of the element,
\item $name \in Names$ is a string description of the element,
\item $decomptype\in DecompTypes$ refers to the type of decomposition.\footnote{Note that, like in the metamodel, the decomposition type is defined on the element.}
\end{itemize}
A set of intentional elements is denoted by $IE$.
\end{definition}

The definition above is sufficient to capture all intentional elements (IEs) used in GRL. %In the next definition, we present some simplifying notational conventions.
%\begin{definition}[Notation]
%\label{def:notation}
%We adopt three conventions simplifying our notation:
%\begin{itemize}
%\item
%We refer to the element of a tuple using the dot (".") notation. That is, we may for instance refer to the id, type, name and decomposition type of an $IE$ with respectively $ie.id$, $ie.type$, $ie.name$, and $ie.decomptype$.
%\item 
%We refer to a set of elements with the same id $i$  using the $i$ subscript on the set. For instance a set of IEs with id $i$ is denoted by $IE_i$, and if this is a single element, we denote it by $ie_i$\footnote{In GRL, there always exists at most one element for every id (see Def.~\ref{def:grl-model}), in RationalGRL, however, this condition does not hold (see Def.~\ref{def:rationalgrl-model}).} For instance, we may refer to the intentional element $ie = (0, Goal, \text{Make profit}, AND)$ with $ie_0$ and write $ie_0.type = Goal$, $ie_0.name = \text{Make profit}$, and $ie_0.decompositiontype = AND$. 
%\item
%We can also refer to intentional elements of a specific type simply by $type_{id}$. For instance, we can abbreviate the element in the previous item with $goal_0$ and write $goal_0.name = \text{Make profit}$, and $goal_0.decomptype = AND$.
%\end{itemize}
%\end{definition}
Throughout this section we use the example in Figure~\ref{fig:example-small}. This example contains various IEs which can be formalized using Definition~\ref{def:ie}, for example, 
$(1, Resource, \text{Car objects}, AND)$, $(2, Softgoal,$ Realistic simulation$, AND)$ and $(5, Goal,$ Generate cars$, XOR)$.

%Using the short-hand notation defined in Def.~\ref{def:notation}, we can make the following statements:
%\begin{itemize}
%\item $goal_4.name = \text{Generate cars}$
%\item $goal_4.decomptype = XOR$
%\item $resource_1.name = \text{Car objects}$
%\end{itemize}
%\begin{figure}[ht]
%\centering
%\includegraphics[width=\columnwidth]{img/Example1-new.pdf}
%\caption{Example GRL model (reprint of Fig.~\ref{fig:example-small})}
%\label{fig:example-small2}
%\end{figure} 
We now define actors.

\begin{definition}[Actor]
\label{def:actor}
An actor is a tuple $act=(id,type, name)$, where:
\begin{itemize}
\item $id\in\mathbb{N}$ is the identifier of the actor, 
\item $type = Actor$ states that this tuple is an actor.
\item $name\in Names$ is a string description of its name.
\end{itemize}
A set of actors is denoted by $Act$.
\end{definition}
We can formalize the actor in our example (Figure~\ref{fig:example-small}) as $(0,Actor,\text{Traffic Simulator})$.%, and we can for instance state $act_0.name = \text{Traffic Simulator}$.

The relation between actors and their intentional element is formalized as follows. 

\begin{definition}[Actor-IE Relations]
\label{def:act-ie-relation}
An \emph{Actor-IE relation} is a tuple $(type, i, j)$, where:
\begin{itemize}
%\item $id\in\mathbb{N}$ is the identifier of the relation, 
\item $type = ActIE$ states that this tuple is an Actor-IE relation,
\item $i\in\mathbb{N}$ is an id of an actor,
\item $j\in\mathbb{N}$ is an id of an IE.
\end{itemize}

A set of Actor-IE relations is denoted by $R_{ActIE}$.
\end{definition}

Note that Actor-IE relations do not have an $id$: they are themselves not elements of a GRL model but rather indicate relations between elements of a GRL model (i.e., intentional elements and actors). The relation between the \emph{Traffic Simulator} actor ($id = 0$) and the \emph{Car objects} resource ($id=1$), for example, can be formalized as $(ActIE, 0, 1)$.

At this point we have defined all intentional elements in GRL and a containment relation between actors and intentional elements. We now turn to the GRL links.

\begin{definition}[GRL Link]
\label{def:link}
A \emph{GRL link} is a tuple $link = (id,type,src,dest)$, where:
\begin{itemize}
\item $id\in \mathbb{N}$ is the unique identifier of the link,
\item $type\in LinkTypes$ specifies the type of the link, 
\item $src\in \mathbb{N}$ is the identifier of the source IE,
\item $dest\in \mathbb{N}$ is the identifier of the destination IE.
\end{itemize}

A set of links is denoted by $Link$.
\end{definition}

%\begin{definition}[Notation]
%\label{def:notation2}
%\begin{itemize}
%\item
%We can refer to links of a specific type simply by $type_{id}$. For instance, we can abbreviate the link $(0,PosContr,2,3)$ with $posContr_0$. 
%\item 
%We can refer to a set of positive contribution links, negative contribution links, dependency links, and decomposition link with respectively $PosContr, NegContr, Dep,$, and $Decomp$. 
%\item
%We can refer to a set of contribution links that can be either positive or negative simply with $Contr$.
%\end{itemize}
%\end{definition}

Similar to IEs, links have identifiers as well. Some of the links in our example (Figure~\ref{fig:example-small}) are $(8, Dep, 4, 1)$, $(9, PosContr, 5, 2)$ and $(11, Decomp, 5, 4)$.

%Let us next provide some examples of the notation of Definition~\ref{def:notation2}:
%\begin{itemize}
%\item $dep_7 = (7, Dep, 4, 1)$
%\item $Decomp = \{(10, Decomp, 5, 4), (11, Decomp, 6, 4)\}$
%\item $PosContr = \{(8, PosContr, 5, 2)\}$
%\item $Contr = \{(8, PosContr, 5, 2), (9, NegContr, 5, 3)\}$
%\end{itemize}

We can now form GRL models as follows.

\begin{definition}[GRL Model]
\label{def:grl-model}
A \emph{GRL model} $GRL=(IE, Act, R_{ActIE}, Link)$ consists of:
\begin{itemize}
\item A set $IE$ of intentional elements (Def.~\ref{def:ie}),
\item A set $Act$ of actors (Def.~\ref{def:actor}),
\item A set $R_{ActIE}$ of Actor-IE relations (Def.~\ref{def:act-ie-relation}),
\item A set $Link$ of GRL links (Def.~\ref{def:link}).
\end{itemize}
\end{definition}

The full specification of Figure~\ref{fig:example-small} is as follows. 

\begin{flalign*}
&IE=&\{&(1, Resource, \text{Car objects}, AND),&\\
&   &  &(2, Softgoal, \text{Realistic simulation}, AND),&\\
&   &  &(3, Softgoal, \text{Simple design}, AND),&\\
&   &  &(4, Softgoal, \text{Easy to use}, AND),&\\
&   &  &(5, Goal, \text{Generate cars}, XOR),&\\
&   &  &(6, Task, \text{Create new cars}, AND),&\\
&   &  &(7, Task, \text{Keep same cars}, AND)\}&\\
&Act=& &\{(0, Actor, \text{Traffic Simulator})\}&\\
&R_{ActIE}=& &\{(ActIE, 0, 1), \ldots, (ActIE,0,7)\}&\\
&Link=&\{&(8, Dep, 4, 1)&\\
&     & &(9, PosContr, 6, 2)&\\
&     & &(10, NegContr, 6, 3)&\\
&     & &(11, Decomp, 6, 5)&\\
&     & &(12, Decomp, 7, 5)\}&\\
\end{flalign*}

Definition \ref{def:grl-model} only sums up the elements of the model and not the constraints that make a \emph{valid} GRL model. We will do so in the next definition. Note that in this definition, we use a subscript notation to refer to an element with a specific id. That is, $IE_i, Link_j$, and $Act_k$ refer respectively to the IE with id $i$, the link with id $j$, and the actor with id $k$.

\begin{definition}[Valid GRL Model]
\label{def:valid-grl-model}
A GRL model $GRL=(IE, Act, R_{ActIE}, Link)$ (Def.~\ref{def:grl-model}) is a \emph{valid GRL model} iff the following conditions are satisfied:
\begin{enumerate}
\item ids are globally unique across IEs, Links, and Actors, i.e., let $X,Y\in \{IE,Act, Link\}$. For all $X_i$ and $Y_j$: if $i=j$ then $X=Y$ and $X_i=Y_j$.
\item Intentional elements of actors exist: $\forall (ActIE, i,j)\in R_{ActIE}: \exists act_i \in Act \wedge \exists ie_j \in IE$.
\item An intentional element belongs at most to one actor: $\forall ie_i\in IE: |\{(ActIE,j,i)\in R_{ActIE}\}| \le 1$.
\item Links connect intentional elements: $\forall (i,type, j,k)\in Link: \{ie_j,ie_k\}\subseteq IE$.
\end{enumerate}
\end{definition}

Let us briefly verify that our previous formalization of Figure~\ref{fig:example-small} satisfies all the constraints of Definition~\ref{def:valid-grl-model}:
\begin{enumerate}
\item All elements in the formalization have different ids, so this constraint is satisfied.
\item $R_{ActIE}$ contains one element for each IE with id $i$, so this constraint is satisfied as well. Note that, in line with the GRL metamodel, $R_{ActIE}$ does not relate links with actors. 
\item There is one actor ($id=0$) and this is the only actor that appears in $R_{ActIE}$ so this constraint is satisfied.
\item All links connect IEs: the contribution links connect elements with ids 2, 3, and 6, which are all IEs; the decomposition links connect elements with ids 5, 6, and 7, which are all IEs; and the dependency link connects id 1 with 4, which are both IEs.
\end{enumerate}

\subsection{Formal specification of RationalGRL}
\label{sect:formalframework:rationalgrl}

In order to develop a logical framework for RationalGRL, we extend the GRL logical framework of the previous section by adding two elements (see Figure~\ref{fig:rationalgrllegend}), namely \emph{generic arguments} and \emph{attack links}. We illustrate the new elements using the RationalGRL model in Figure~\ref{fig:example-small3}, which is an extension of Figure~\ref{fig:example-small}. 

\begin{figure}[b]
\centering
\includegraphics[width=\columnwidth]{img/Example1-new-attack.pdf}
\caption{Example RationalGRL model (extension of Fig.~\ref{fig:example-small})}
\label{fig:example-small3}
\end{figure} 

\begin{definition}[Generic Argument]
\label{def:generic-argument}
A generic argument is a tuple $ga=(id, type, name)$, where:
\begin{itemize}
\item $id\in \mathbb{N}$ is the identifier of the generic argument,
\item $type = GenArg$ states that the tuple is a generic argument,
\item $name\in Names$ is a string description of its name.
\end{itemize}
The set of generic arguments is denoted by $AE$.
\end{definition}

So a generic argument is any element in the RationalGRL model that is not an intentional element or an actor. In the example, some of the generic arguments are $(13, GenArg, \text{Redundant})$ and $(14,$ $GenArg,$ Necessary$)$. Note that a constraint of a GRL model (Definition~\ref{def:grl-model}) is that GRL links should connect IEs, which means that in generic arguments cannot be connected with GRL links. 
 
We can now define \emph{arguments}. 

\begin{definition}[Argument]
\label{def:argument}
An argument $A$ is on of the following tuples:
\begin{itemize}
\item An intentional element $ie$ (Def.~\ref{def:ie}), 
\item An actor $act$ (Def.~\ref{def:actor}),
\item An Actor-IE relation $r_{ActIE}$ (Def.~\ref{def:act-ie-relation}), 
\item A link $link$ (Def.~\ref{def:link}),
\item A generic argument $ga$ (Def.~\ref{def:generic-argument}).
\end{itemize}
\end{definition}

This definition captures the specification in the RationalGRL metamodel (Figure~\ref{fig:metamodel}) in which the class \textsf{Argument} is a superclass of \textsf{GenericArgument} and \textsf{GRLModelElement}. In sum, we define an argument simply as any one of the GRL elements or links, or a generic argument. 

Examples of arguments in Figure~\ref{fig:example-small3} are $A_5 = (5,$ $Goal,$ Generate cars$, AND)$, $A'_5 = (5, Goal,$ Generate cars$, XOR)$, $A_4 = (4, Softgoal,$ Easy to use$, AND)$ and $A_{13}=(13,$ $ArgElem,$ Redundant$)$. Note that the two arguments $A_5$ and $A'_5$ show an important difference between RationalGRL models and valid GRL models: While a valid GRL model disallows multiple elements with the same identifier (Definition~\ref{def:valid-grl-model}, condition 1), RationalGRL models do not enforce this restriction. This is because it is possible to create multiple arguments for the same element, where each argument contains different content for the same element. However, the set of \emph{accepted} elements in a RationalGRL should all have unique identifier (see Definition~\ref{def:valid-rationalgrl-model}).

\begin{definition}[Attack Link]
\label{def:link:attack}
Given a set of arguments $Arg$, an attack link $att=(A_i,A_j)$ means:
\begin{itemize}
\item $A_i\in Arg$ is the argument performing that attack,
\item $A_j\in Arg$ is the argument being attacked.
\end{itemize}
A set of attack links is denoted by $Att$.
\end{definition}

As an example, take the arguments $A_5$ for the goal `Generate cars' ($AND$ decomposition), $A'_5$ for the goal `Generate cars' ($XOR$ decomposition), $A_4$ for the softgoal `Easy to use' and the generic argument $A_{13}$ (`Redundant') against the softgoal (`Easy to use'). Given these arguments there are two attack links (Figure~\ref{fig:example-small3}), namely $(A'_{5},A_{5})$ and $(A_{13}, A_{4})$.

We now define a RationalGRL model.

\begin{definition}[RationalGRL Model]
\label{def:rationalgrl-model}
A \emph{RationalGRL model} $RatGRL=(Arg, Att)$ consists of a set of arguments $Args$ (Def.~\ref{def:argument}) and a set of attack links $Att$ (Def.~\ref{def:link:attack}).
\end{definition}

The definition of a RationalGRL model collects all the previously defined GRL and RationalGRL elements in a single definition. For completeness, we now provide the full specification of Figure~\ref{fig:example-small3}. Let us first enumerate all the arguments used in this example:

\begin{flalign*}
A_0 = &(0, Actor, \text{Traffic simulator})&\\
A_1 = &(1, Task, \text{Car objects}, AND),&\\
A_2 = &(2, Softgoal, \text{Realistic simulation}, AND),&\\
A_3 = &(3, Softgoal, \text{Simple design}, AND),&\\
A_4 = &(4, Softgoal, \text{Easy to use}, AND),&\\
A_5 = &(5, Goal, \text{Generate cars}, AND),&\\
A'_5 = &(5, Goal, \text{Generate cars}, XOR),&\\
A_6 = &(6, Task, \text{Create new cars}, AND),&\\
A_7 = &(7, Task, \text{Wrap cars}, AND),&\\
A'_7 = &(7, Task, \text{Keep same cars}, AND),&\\
A_8 = &(8, Dep, 4, 1),&\\
A_9 = &(9, PosContr, 6, 2),&\\
A_{10} = &(10, NegContr, 6, 3),&\\
A_{11} = &(11, Decomp, 6, 5),&\\
A_{12} = &(12, Decomp, 7, 5),&\\
A_{13} = &(13, GenArg, \text{Redundant}),&\\
A_{14} = &(14, GenArg, \text{Necessary}),&\\
A_{15} = &(15, GenArg, \text{No effect}),&\\
A_{16} = &(16, GenArg, \text{Changes traffic flow}),&\\
A_{17} = &(ActIE,0,1),\ldots,A_{23}=(ActIE,0,7)&\\
\end{flalign*}
This model is then formalized as $RatGRL=(Arg, Att)$ where:
\begin{flalign*}
Arg = &\{A_0,\ldots,A_5, A'_5,A_6,A_7,A'_7\ldots,A_{23}\}\\
Att= &\{(A'_{5},A_5), (A'_{7},A_{7}), (A_{13},A_{4}), (A_{14},A_{13}),\\
     & (A_{15},A_{9}),(A_{16},A_{15})\}\\
\end{flalign*}

%All the arguments and the attack relations of this RationalGRL model are shown in Figure~\ref{fig:example-small4}. Note the arguments for Actor-IE containment (arguments $A_{12}$ to $A_{17}$) have been omitted from this figure for readability. It can be read from the figure that two arguments are currently rejected, namely $A_4$ and $A_{19}$. However, we did not yet make precise how exactly this is computed. We will do so in the following definitions.

%\begin{figure}[ht]
%\centering
%\includegraphics[width=\columnwidth]{img/Example1-new-arguments}
%\caption{Argumentation framework of RationalGRL model from Fig.~\ref{fig:example-small3}}
%\label{fig:example-small4}
%\end{figure} 

From Figure~\ref{fig:example-small3}, it can be read from that arguments $A_5$, $A_7$, $A_{13}$ and $A_{15}$ are currently disabled (rejected). However, we did not yet make precise how exactly this is computed. We will do so in the following definitions.
\begin{figure}[b]
\centering
\includegraphics[]{img/Example1-new-arguments.pdf}
\caption{Example argumentation framework, subset of the RationalGRL model from Figure~\ref{fig:example-small3}.}
\label{fig:goalmodeling:arg2}
\end{figure}
In order to compute when an argument is accepted and when not we use argumentation semantics.  We use the standard approach by Dung~\cite{Dung1995}. 

\begin{definition}[Argumentation Framework]
\label{def:argumentation-framework}
An \emph{argumentation framework} $AF=(Arg,Att)$ consists of a set of arguments $Arg$ and a set of attack relations $Att\subseteq Arg\times Arg.$
\end{definition}

Note that the definition of a RationalGRL model (Definition~\ref{def:rationalgrl-model}) is exactly the same as the definition for an argumentation framework. This allows us to use the following definitions directly.

\begin{definition}[Attack, conflict-freeness, defense, admissibility, preferred extension] \label{def:semantics}Suppose an argumentation framework $AF=(Arg,Att)$, two sets of arguments $S\cup S'\subseteq Arg$, and some argument $A\in Arg$. We say that
\begin{itemize}
\item $S$ \emph{attacks} $A$ iff some argument in $S$ attacks $A$,
\item $S$ \emph{attacks} $S'$ iff some argument in $S$ attacks some argument in $S'$,
\item $S$ is \emph{conflict-free} iff it does not attack itself,
\item $S$ \emph{defends} $A$ iff for each $B$ such that $B$ attacks $A$, $S$ attacks $B$,
\item $S$ is \emph{admissible} iff $S$ is conflict-free and defends each argument in it.
\item $S$ is a \emph{preferred extension} iff it is a maximal (w.r.t. set inclusion) admissible set.
\end{itemize}
\end{definition}

Let us explain these definitions using the example argumentation framework in Figure~\ref{fig:goalmodeling:arg2}, which is a subset of our RationalGRL model from Figure~\ref{fig:example-small3}, containing only arguments $A_4, A_{5},A'_{5},A_{13}$, and $A_{14}$. In this example, there are five admissible sets: $\{A'_{5}\}$, $\{A_{14}\}$, $\{A'_{5},A_{14}\}$, $\{A_4, A_{14}\}$ and $\{A_4, A'_5, A_{14}\}$. In the admissible sets that contain both $A_{4}$ and $A_{14}$, we say that $A_{14}$ defends $A_4$ against its attacker $A_{13}$. Sets containing both $A_{5}$ and $A'_{5}$, or both $A_{13}$ and either $A_4$ or $A_{14}$ are not conflict free. Sets containing $A_5$ are not admissible, as they do not defend $A_5$. Similarly, sets containing $A_4$ but not $A_{14}$ are not admissible as they do not defend themselves against $A_{13}$. 

The notion of admissible sets gives rise to various possible extensions of an argumentation framework; in this article, we use the preferred extension. In Figure~\ref{fig:goalmodeling:arg2}, there is one preferred extension, namely $\{A_4, A'_5, A_{14}\}$. It is possible to have multiple preferred extensions in cases where the argumentation framework contains cycles. A simple example of such an argumentation framework is shown in Figure~\ref{fig:goalmodeling:arg3}, where arguments $A_x$ and $A_y$ attack each other. There are two preferred extensions, namely $\{A_x\}$ and $\{A_y\}$. Although the algorithms in Section~\ref{sect:algorithms} do not generate mutually attacking arguments such as in Figure~\ref{fig:goalmodeling:arg3}, our formal framework does not explicitly forbid attack cycles in RationalGRL models. We discuss this in more detail in Section~\ref{sect:discussion:futurework}.

\begin{figure}[ht]
\centering
\includegraphics[]{img/example-mutual-attack.pdf}
\caption{Example of a cyclic argumentation framework.}
\label{fig:goalmodeling:arg3}
\end{figure}

The status of individual arguments can now be determined on the basis of the preferred extensions. Recall from the metamodel (Figure~\ref{fig:metamodel}) that an argument can be \emph{acceptable}, \emph{undecided} or \emph{rejected}. 

\begin{definition}[Argument Acceptability] 
\label{def:acceptability}
\begin{itemize}
\item An argument is \emph{acceptable} w.r.t. an argumentation framework $AF$ if it is in every preferred extension of $AF$. 
\item An argument is \emph{undecided} w.r.t. an argumentation framework $AF$ if it is in at least one but not all preferred extensions of $AF$. 
\item An argument is \emph{rejected} w.r.t. an argumentation framework $AF$ if it is in no preferred extension of $AF$. 
\end{itemize}
\end{definition}

Take our two simple examples. In Figure~\ref{fig:goalmodeling:arg2}, we have that arguments $A_4$, $A'_5$ and $A_{14}$ are acceptable and arguments $A_5$ and $A_{13}$ are rejected. In Figure~\ref{fig:goalmodeling:arg3}, both arguments $A_x$ and $A_y$ are undecided. Returning to our larger example from Figure~\ref{fig:example-small3}, we can see that arguments $A_5$, $A_7$, $A_{13}$ and $A_{15}$ are rejected and all the other arguments are acceptable.

\subsection{Translating between RationalGRL and GRL}
\label{sect:formalframework:translation}

Now that we have formalized both a GRL model and a RationalGRL model, we present algorithms to translate between these two models. Both of these two translation algorithms are straightforward, which is result of the fact that the two models are formalized in a very similar way.

\paragraph{GRL to RationalGRL Translation} We start with the translation algorithm from GRL to RationalGRL, which is shown in Algorithm~\ref{alg:translation:to-rationalgrl}. The translation algorithm takes a GRL model $GR=(IE, Act, R_{ActIE}, Link)$ (Definition~\ref{def:grl-model}) as input. On line~\ref{alg:translation:to-rationalgrl:args}, it collects all the elements of the GRL model into a set $Arg$, which represents the set of arguments of the RationalGRL model. On line~\ref{alg:translation:to-rationalgrl:att}, the set of attack links is initialized with an empty set, and the new RationalGRL model $(Arg, Att)$ is returned on line~\ref{alg:translation:to-rationalgrl:return}.

\begin{algorithm}[ht]
  \caption{GRL to RationalGRL Translation}
  \label{alg:translation:to-rationalgrl}
  \begin{algorithmic}[1]
    \Procedure{$ToRationalGRL$}{$GRL$}
    \State $Arg \leftarrow IE\cup Act \cup R_{ActIE}\cup Link$\label{alg:translation:to-rationalgrl:args}
    \State $Att \leftarrow \emptyset$\label{alg:translation:to-rationalgrl:att}
    \State \Return $(Arg, Att)$\label{alg:translation:to-rationalgrl:return}
    \EndProcedure
  \end{algorithmic}
\end{algorithm}

\paragraph{RationalGRL to GRL translation} The translation from a RationalGRL model to a GRL model is given in Algorithm~\ref{alg:translation:to-grl}. First. the arguments are each put in the corresponding GRL component sets. The procedure $ComputeExtensions(Arg,Att)$ (line~\ref{alg:translation:to-grl:extension}) computes the (preferred) extensions of the RationalGRL model (Definition~\ref{def:semantics}). A GRL model is then generated from each of the preferred extensions. This is done by iterating over all arguments in the extension in line~\ref{alg:translation:to-grl:for}. The switch statement on line~\ref{alg:translation:to-grl:switch} then does a case distinction on the type of the arguments -- here, $Arg.type$ refers to the $type$ element in an argument tuple $Arg$ -- and each case ensures the argument is put in the right GRL component. Finally, the algorithm returns a GRL model on line~\ref{alg:translation:to-grl:return}. As an example, compare the RationalGRL model in Figure~\ref{fig:example-small3} to the GRL model in Figure~\ref{fig:example-small}, where the latter is a translation of the former.

\paragraph{Valid RationalGRL model} While we have defined a notion of a \emph{valid GRL model} (Definition~\ref{def:valid-grl-model}), we have not done so for a RationalGRL model yet. We define a RationalGRL model as valid if and only if the RationalGRL to GRL translation results in a valid GRL model. Thus, we do not have to reiterate all conditions on a GRL model, but use the translation algorithm.

\begin{algorithm}[b]
  \caption{RationalGRL to GRL Translation}
  \label{alg:translation:to-grl}
  \begin{algorithmic}[1]
    \Procedure{$ToGRL$}{$RatGRL$}
    \State $Ext \leftarrow ComputeExtensions(Arg,Att)$\label{alg:translation:to-grl:extension}
    \For{$E \in Ext$} 
    \State $IE\leftarrow\emptyset$, $Act\leftarrow\emptyset$, $R_{ActIE}\leftarrow\emptyset$, $Link\leftarrow \emptyset$
    \For{$A\in E$}\label{alg:translation:to-grl:for}
      \Switch{$A.type$}\label{alg:translation:to-grl:switch}
          \Case{$IETypes$}
            \State $IE\leftarrow IE\cup \{A\}$
          \EndCase
          \Case{$Actor$}
            \State $Actor\leftarrow Actor\cup \{A\}$
          \EndCase
          \Case{$ActIE$}
            \State $ActIE\leftarrow ActIE\cup \{A\}$
          \EndCase
          \Case{$LinkTypes$}
            \State $Link\leftarrow Link \cup\{A\}$
          \EndCase
      \EndSwitch
    \EndFor
    \State \Return $(IE,Act,R_{ActIE}, Link)$\label{alg:translation:to-grl:return}
    \EndFor
    \EndProcedure
  \end{algorithmic}
\end{algorithm}

\begin{definition}[Valid RationalGRL Model]
\label{def:valid-rationalgrl-model}
A RationalGRL model $RatGRL = (Arg, Att)$ (Def.~\ref{def:rationalgrl-model})
is a \emph{valid RationalGRL model} iff $ToGRL(RatGRL)$ (Algorithm~\ref{alg:translation:to-grl}) is a valid GRL model (Def.~\ref{def:valid-grl-model}).
\end{definition}


\subsection{Algorithms for argument schemes and critical questions}
\label{sect:algorithms}

We have now formalized a \emph{static} representation of the RationalGRL framework. In this section we formalize the \emph{dynamics} by developing algorithms for applying argument schemes and critical questions. These algorithms are applied to RationalGRL models (Definition~\ref{def:rationalgrl-model}) and produce new arguments and attack relations. 

As discussed in Section~\ref{sect:overview}, the argument schemes and critical questions of Table~\ref{table:argument-schemes} all lead to one of three operations: \textsf{INTRO}
introduces a new RationalGRL element and \textsf{DISABLE} creates a new argument that attacks another argument. The \textsf{REPLACE} operation is can be seen as a combination of \textsf{INTRO} and \textsf{DISABLE}: a new argument corresponding to a GRL element is added and this new argument attacks a previous version of the element or link. We will discuss all three operations in separate sections below.

In all of the following algorithms, we assume that:
\begin{itemize}
\item The algorithms are applied to some valid RationalGRL model $RatGRL$ (Definition~\ref{def:valid-rationalgrl-model}),
\item The procedure $mintId()$ generates a new unique id.
\end{itemize}

\subsubsection{INTRO algorithms}
\label{sect:formalframework:intro}

The following arguments schemes and critical questions of Table~\ref{table:argument-schemes} fall into this category:
\begin{itemize}
\item AS0-AS12
\item CQ5b, CQ6b, CQ6c, CQ7b, CQ9, CQ10b
\end{itemize}

These type of algorithms are short, and consist simply of adding an argument for the element that is being added. 

\begin{algorithm}[h]
  \caption{AS0: $a$ is an actor}\label{alg:add-actor}
  \begin{algorithmic}[1]
    \Procedure{$AS_0$}{$n$}
    \State $A \leftarrow (mintId(), Actor, a)$ \label{alg:as0:1}
    \State $Arg\leftarrow Arg \cup \{A\}$\label{alg:as0:2}
    \EndProcedure
  \end{algorithmic}
\end{algorithm}

Algorithm~\ref{alg:add-actor} takes one argument, namely the name of the actor $a$. On line~\ref{alg:as0:1} of the algorithm, a new (unique) id is minted as the identifier of the new actor, which is assigned with its corresponding name to the variable $actor$. On line~\ref{alg:as0:2} the actor is added as a new argument. As an example, we can formalize the addition of actor ``Traffic Simulator'' to an empty RationalGRL model (i.e. $Arg=\emptyset$ and $Att=\emptyset$) as executing the algorithm $AS_0$(Traffic simulator), which results in a RationalGRL model with argument $A_0 = (0, Actor, \text{Traffic simulator})\in Arg$.

\begin{algorithm}[h]
  \caption{AS1: Actor with id $i$ has resource $R$}\label{alg:add-resource}
  \begin{algorithmic}[1]
    \Procedure{$AS_1$}{$i, R$}
    \State $res\_id\gets mintId()$\label{alg:add-resource:id}
    \State $A_1\leftarrow (res\_id, Resource, R, AND)$\label{alg:add-resource:arg1}
    \State $A_2\leftarrow (ActIE, i, res\_id)$\label{alg:add-resource:arg2}
    \State $Arg\gets Arg\cup \{A_1,A_2\}$\label{alg:add-resource:add-arg}
    \EndProcedure
  \end{algorithmic}
\end{algorithm}

In Algorithm~\ref{alg:add-resource}, we have slightly reworded the original AS1 from Table~\ref{table:argument-schemes} to ``Actor with id $i$ has resource $R$''. Furthermore, we assume actor $a$ already exists when Algorithm~\ref{alg:add-actor} is run. The algorithm itself runs as follows: on line~\ref{alg:add-resource:id}, a unique id is assigned to variable $res\_id$ (``resource identifier''). On line~\ref{alg:add-resource:arg1}, an argument for a resource with name $R$ and identifier $res\_id$ is assigned to $A_1$. On line~\ref{alg:add-resource:arg2}, a second argument $A_2$ is created, which is an argument for an Actor-IE relation (Def.~\ref{def:act-ie-relation}) between the input actor $i$ and the newly created element $res\_id$. Finally, on line~\ref{alg:add-resource:add-arg}, the arguments are added to the set of arguments $Arg$.

As an example, consider the addition of resource ``Car objects'' to the RationalGRL model we just discussed (where $A_0 = (0, Actor, \text{Traffic simulator})$). Adding the resource can then be formalized as executing $AS1(0, \text{Car objects})$, resulting in the addition of two arguments $(1, Resource, \text{Car object}, AND)$ and $(ActIE, 0, 1)$ which are added to the set of arguments $Arg$.

Since arguments schemes AS2-AS4 are very similar to AS1, we omit these algorithms here.

\begin{algorithm}[h]
  \caption{AS5: Goal with id $i$ decomposes into task $T$}\label{alg:add-decomp}
  \begin{algorithmic}[1]
    \Procedure{$AS_5$}{$i, T$}
    \State $task\_id\gets mintId()$\label{alg:add-decomp:task-id}
    \State $A_1\leftarrow (task\_id, Task, T, AND)$\label{alg:add-decomp:arg1}
    \State $A_2\leftarrow (mintId(), Decomp, i, task\_id)$\label{alg:add-decomp:arg2}
    \State $Arg\gets Arg\cup \{A_1,A_2\}$\label{alg:add-decomp:add-arg}
    \EndProcedure
  \end{algorithmic}
\end{algorithm}

Similar to the previous algorithm, we have slightly reworded critical question AS5 in Algoritm~\ref{alg:add-decomp}. We assume that a goal $G$ exists already with identifier $i$, and that some new task with name $T$ is a decomposition of $G$. In the algorithm, on line~\ref{alg:add-decomp:task-id} a unique identifier is created for the task, which is created on line~\ref{alg:add-decomp:arg1}. On line~\ref{alg:add-decomp:arg2} an argument is created for the decomposition link $(mintId(), Decomp, i, task\_id)$ (Def.~\ref{def:link}, going from the existing goal with identifier $i$ to the new task with identifier $task\_id$. On line~\ref{alg:add-decomp:add-arg} the two arguments are added to the set of arguments $Arg$.

As an example, suppose we are adding the decomposition of goal ``Generate cars'', expressed as argument $A_5 = (5,$ $Goal,$ Generate cars$, AND)$, into task ``Keep same cars'', and suppose we have the following argument for the goal: . Adding the decomposition can be formalized as executing $AS_5(5, \text{Keep same cars})$, and results in two new arguments: $A_7 = (7, Task, \text{Keep same cars}, AND)$ and $A_{12} = (12, Decomp, 7, 5)$, which are both added to the set of arguments $Arg$.

The remaining argument schemes AS6-AS12 are all similar to algorithms~\ref{alg:add-actor}-\ref{alg:add-decomp} and have been omitted here.

The critical questions of type \emph{INTRO} are very similar as well, with one exception: they require an answer. For instance, suppose CQ5b: ``Does goal $G$ decompose into other tasks?'' is answered with: ``Yes, namely into task $T$''. In this case, we simply obtain an instantiation of argument scheme AS5: ``Goal $G$ decomposes into task $T$'', which can be executed with Algorithm~\ref{alg:add-decomp}. This is the same for all the other critical questions of type \emph{INTRO}. Therefore, we have omitted them here as well.

\subsubsection{DISABLE algorithms}
\label{sect:formalframework:disable}
As discussed before, algorithms of type \emph{DISABLE} consist of adding a new argument attacking an existing argument. The argument that is added is itself not an argument for a GRL element or link, but rather a generic argument (Definition~\ref{def:generic-argument}).

In all of these algorithms, we assume the critical question is answered as indicated in the right-most column of Table~\ref{table:argument-schemes}. For instance, for critical question CQ0 ``Is the actor relevant?'', we assume it is answered with ``No''.

\begin{algorithm}[h]
  \caption{CQ0: Is actor with id $i$ relevant? No}\label{alg:actor-not-relevant}
  \begin{algorithmic}[1]
    \Procedure{$CQ_0$}{$i$}
    \State $A \leftarrow (mintId(),GenArg,CQ0)$\label{alg:actor-not-relevant:genarg}
    \State $Arg\leftarrow Arg \cup \{A\}$\label{alg:actor-not-relevant:genarg2}
    \For{$A_j\in \{A=(i,Actor,n)\mid A\in Arg\}$}\label{alg:actor-not-relevant:for}
      \State $Att \leftarrow Att \cup \{(A,A_j)\}$\label{alg:actor-not-relevant:att}
    \EndFor
    \EndProcedure
  \end{algorithmic}
\end{algorithm}

Algorithm~\ref{alg:actor-not-relevant} is executed when critical question CQ0 is answered affirmatively (i.e., with ``No''). First, on lines~\ref{alg:actor-not-relevant:genarg} and~\ref{alg:actor-not-relevant:genarg2}, an argument is created for the critical question and added to the set of arguments $Arg$. Since this argument is not an argument for a GRL element or link, it is formalized as a \emph{generic counterargument} $(mintId(), GenArg, CQ0)$ (Def.~\ref{def:generic-argument}). The for loop starting at line~\ref{alg:actor-not-relevant:for} then iterates over all arguments for actors, where $i$ is the id of the actor that is no longer relevant. The reason why there could be multiple of such actors is that the actor can be refined by an algorithm of type \emph{REPLACE}. We will explain this in more detail in the example below. Then, on line~\ref{alg:actor-not-relevant:att}, an attack link is created from the generic argument $A$ that is created to the argument for the actor $A_j$. After executing the algorithm, all existing arguments for the actor with identifier $i$ are attacked by a newly created argument $A$.

Consider for example the RationalGRL model in Figures~\ref{fig:examples:relevant-actor}, which consists of an actor and a generic counterargument. Let us reconstruct this model using the application of argument schemes and critical questions to an initially empty RationalGRL model ($Arg=\emptyset$ and $Att=\emptyset$). The algorithm $AS_0(\text{Development team})$ is executed, which results in $A_1=\{(0,Actor,\text{Development team}\})$. After this, critical question CQ0: ``Is the actor Development team relevant?'' is answered with ``No, because the professor develops the software''. We can formalize this as executing algorithm $CQ_0(0)$ (Algorithm~\ref{alg:actor-not-relevant}), which results in adding a generic counter argument $A_2=(1, GenArg, \text{CQ0: Professor develops software})$ and an attack link $(A_2,A_1)$. The RationalGRL model now consists of two arguments, and that the preferred extension is $\{A_2\}$, which means that the resulting GRL model is empty, because generic arguments are not translated to GRL.

Critical questions C1-CQ3 are all very similar to CQ0 and have therefore been omitted here.

\begin{algorithm}[h]
  \caption{CQ5a: Does the goal with id $g\_id$ decompose into task with id $t\_id$? No}\label{alg:no-decomp}
  \begin{algorithmic}[1]
    \Procedure{$CQ5a$}{$g\_id,t\_id$}
    \State $A \leftarrow (mintId(),GenArg,CQ5a)$\label{alg:no-decomp:genarg}
    \State $Arg\leftarrow Arg \cup \{A\}$\label{alg:no-decomp:genarg2}
    \For{$A_j\in \{(k,Decomp,g\_id,t\_id)\in Arg\}$}\label{alg:no-decomp:for}
      \State $Att \leftarrow Att \cup \{(A,A_j)\}$\label{alg:no-decomp:att}
    \EndFor
    \EndProcedure
  \end{algorithmic}
\end{algorithm}

Algorithm~\ref{alg:no-decomp} is structurally very similar to Algorithm~\ref{alg:actor-not-relevant}, with the only difference that instead of iterating over actors, we now iterate over decomposition links. This is done in line~\ref{alg:no-decomp:for}, where we iterate over all decomposition links with source id $g\_id$ and target id $t\_id$ (Definition~\ref{def:link}). This means we iterate over all decomposition links from the goal to the task, and we attack all arguments for these links on line~\ref{alg:no-decomp:att} with the newly created argument.

Almost all of the remaining critical questions of type DISABLE are similar in structure. CQ11 (``Is the element relevant/useful'') is slightly different since the attack element is not of a specific type, but is simply any GRL element. However, the resulting algorithm is very similar to the previous two and has therefore been omitted. The only other algorithm of type \emph{DISABLE} that we discuss is $Att$.

\begin{algorithm}[b]
  \caption{Att: Generic counter-argument on arguments $A_1,\ldots,A_n$ with name $N$}\label{alg:generic}
  \begin{algorithmic}[1]
    \Procedure{$Att$}{$A_1,\ldots,A_n$, $N$}
    \State $A \leftarrow (mintId(),GenArg, N)$\label{alg:generic:arg}
    \State $Arg\leftarrow Arg \cup \{A\}$\label{alg:generic:addarg}
    \For{$A_j\in \{A_1,\ldots,A_n\}$}\label{alg:generic:for}
      \State $Att \leftarrow Att \cup \{(A,A_j)\}$\label{alg:generic:att}
    \EndFor
    \EndProcedure
  \end{algorithmic}
\end{algorithm}

Algorithm~\ref{alg:generic} can be regarded as the most general way of providing counterarguments to arguments. In all of the previous \emph{DISABLE} algorithm, the attack was on a specific type of argument, for instance an argument for an actor or an argument for a decomposition. In this algorithm, however, \emph{any} set of arguments can be attacked by a new argument.

Let us reconsider Figure~\ref{fig:examples:relevant-actor}, which we previously formalized as a RationalGRL model where $Arg=\{A_1,A_2\}$, $A_1 = \{(0,Actor,\text{Development team}\})$, $A_2= (1,$ $GenArg,$ CQ0: Professor develops software$)$ and $Att=(A_2,A_1)$. Suppose we execute algorithm $Att(\{A_2\},$ Professor does not develop software$)$. This results in $A_3=(2, GenArg,$ CQ0: Professor does not develop software$)$ and a new attack link $(A_3,A_2)$ (Figure~\ref{fig:examples:relevant-actor2}). Since the argument for the goal is now part of the preferred extension, it does show up in the resulting GRL model after the translation of Algorithm~\ref{alg:translation:to-grl}.

\begin{figure*}[b]
\centering
\includegraphics{img/methodology.pdf}
\caption{The RationalGRL Methodology}
\label{fig:rationalgrl-methodology}
\end{figure*}

\subsubsection{REPLACE algorithms}
\label{sect:formalframework:replace}

Recall that \emph{REPLACE} algorithms create a new argument that attacks all arguments for an existing element.

\begin{algorithm}[h]
  \caption{CQ5c: Is the decomposition type of element $ie_i$ correct? No, it should be $X$ }\label{alg:replace1}
  \begin{algorithmic}[1]
    \Procedure{$CQ_{Decomp}$}{$ie_i, X$}
    \State $A \leftarrow ie_i$\label{alg:replace1:arg}
    \State $A.decomptype\leftarrow X$\label{alg:replace1:decompchange}
    \State $IEArgs\leftarrow IE_i\subseteq  Arg$\label{alg:replace1:ieargs}
    \For{$\{(A_i,A_j)\in Att\mid A_j\in IEArgs\}$}\label{alg:replace1:for1}
      \State $Att\leftarrow (A_i,A)$
    \EndFor
    \For{$\{(A_i,A_j)\in Att\mid A_i\in IEArgs\}$}\label{alg:replace1:for2}
      \State $Att\leftarrow (A,A_j)$
    \For{$A_i\in IEArgs$}\label{alg:replace1:for3}
      \State $Att \leftarrow Att \cup \{(A,A_i)\}$\label{alg:replace1:att}
    \EndFor
    \EndFor
    \State $Arg\leftarrow Arg \cup \{A\}$\label{alg:replace1:addarg}
    \EndProcedure
  \end{algorithmic}
\end{algorithm}

While the original critical question CQ5c is specific to the decomposition between a goal and a task, Algorithm~\ref{alg:replace1} is more generally applicable to the decomposition type of any IE, since all IEs have a decomposition type (Definition~\ref{def:ie}). 

Let us go through this algorithm step by step. On line~\ref{alg:replace1:arg}, a new argument $A$ is created which is identical to original IE. On line~\ref{alg:replace1:decompchange} the decomposition type of the argument is changed to $X$ -- here, $Arg.decomptype$ refers to the $decomptype$ element in an argument tuple $Arg$. On line~\ref{alg:replace1:ieargs}, the set $IEArgs$ is assigned with all existing arguments for the input IE -- here $IE_i$ is the set of all IEs with id $i$. The first two for loops on respectively lines~\ref{alg:replace1:for1} and~\ref{alg:replace1:for2} ensure that all attack links that existing from and to the previous versions of the IE are also carried over to the new argument $A$. Then, in the third for loop on line~\ref{alg:replace1:for3}, we add attack links from the argument that has just been created to all existing arguments for the IE. This ensures that all previous version of the IE are not part of the preferred extension, and as a result do not show up in the resulting GRL model. Finally, one line~\ref{alg:replace1:addarg}, the new argument is added to the set of arguments.

As an example, take Figure~\ref{fig:examples:decomposition}. The initial RationalGRL model (before applying CQ10b), can be formalized as follows: $RatGRL=(Arg,Att)$, with $Arg=\{A_1,A_2,A_3,A_4,A_5\}$ and $Att=\emptyset$, where:
\begin{itemize}
\item $A_1=(0,Goal,\text{Simulate traffic},AND)$
\item $A_2=(1,Task,\text{Dynamic simulation},AND)$
\item $A_3=(2,Task,\text{Static simulation},AND)$
\item $A_4=(3,Decomp,1,0)$
\item $A_5=(4,Decomp,2,0)$
\end{itemize}

Suppose algorithm $CQ_{Decomp}(0, XOR)$ is executed for this RationalGRL model. On lines~\ref{alg:replace1:arg} and \ref{alg:replace1:decompchange} a new argument $A_6=(0,Goal,\text{Simulate traffic},OR)$ is created. On line~\ref{alg:replace1:ieargs}, $IEArgs$ contains all arguments for element 0, which is $\{A_1\}$ (note this does not yet include argument $A_6$, since it is not yet added to the RationalGRL model). The first two for loops on lines~\ref{alg:replace1:for1} and~\ref{alg:replace1:for2} do not do anything in the case of our example, since $A_1$ is itself not attack or does not attack any other argument. The for loop on line~\ref{alg:replace1:for3} ensures all previous versions of the element with id 0 are now attacked, which means in our case the $(A_6,A_1)$ is added to $Att$. Finally, on line~\ref{alg:replace1:addarg} the new argument $A_6$ is added to $Arg$.

The other \emph{REPLACE} algorithms are similar to Algorithm~\ref{alg:replace1}, which can be used directly for CQ10c. For CQ12 we should make a small modification: instead of replacing the decomposition type of the IE, we replace its name. Since this is a very minor modification we have omitted it here.
\section{The RationalGRL Tool}
\label{sect:implementation}
\section{The RationalGRL Methodology and Tool}
\label{sect:methodology+tool}

In previous sections, we have shown how the RationalGRL framework can capture stakeholders discussions, and how interactions between two types of reasoning, practical reasoning and goal modeling, leads to two interlinked models, RationalGRL and GRL models. The previous section contained a formalization of RationalGRL, which forms the starting point of this section. In this section we clarify how practitioners can actually use the RationalGRL framework by proposing a methodology (\textbf{requirement 4}) and discussing a prototype RationalGRL tool (\textbf{requirement 5}).

\subsection{RationalGRL Methodology}
\label{sect:methodology} 

We propose the methodology shown in Figure~\ref{fig:rationalgrl-methodology} to develop a RationalGRL model, a version of which was presented at the 2017 iStar workshop \cite{ghanavatiMethodology}. Here we assume that the initial GRL models have been created based on the requirements specification documents and the discussions of the stakeholders. The rest of the steps are as follows:

\textbf{(1) Instantiate Argument Schemes (AS)} -- In this step, we start from the list of arguments schemes (Table~\ref{table:argument-schemes}). Whilst discussing the requirements, we select schemes from the list and instantiate them to form arguments for GRL IEs or links. In this way we build the GRL model by introducing new GRL elements (\textsf{INTRO}). For example, an argument scheme can be "Goal \emph{G} contributes to softgoal \emph{S}". When an argument scheme is instantiated, it corresponds to an argument for or against part of a goal model.

\textbf{(2) Answer Critical Questions (CQs)} -- After building or modifying the initial GRL model, we ask the relevant critical questions. Since each element in the GRL model corresponds to an instantiated scheme, we can look at Table~\ref{table:argument-schemes}) to see which questions are relevant given our GRL model. For example, for the argument scheme, "Goal \emph{G} contributes to softgoal \emph{S}", there are two critical questions as follows:  \emph{Does the goal contributes to the softgoal?} and \emph{Does the goal contributes to some other softgoals?}. When the analyst answers  a critical question, a new argument scheme may be instantiated. This is done in step (3) when the operation INTRO is executed.

\textbf{(3) Decide on Intentional Elements and their Relationships} -- By answering a critical question, one of the three operations (\textsf{INTRO}, \textsf{DISABLE} or \textsf{REPLACE}) is applied to the GRL model. Any of these operations impact the arguments and corresponding GRL intentional elements, modifying the initial GRL model into a RationalGRL model. Recall that \textsf{INTRO} means that 
a new argument scheme is created. That means, the current argument related to the critical question does not get attacked.  In the case of \textsf{DISABLE}, the intentional element or its related links are disabled in the model. \textsf{REPLACE} introduces a new argument and attacks the original argument at the same time. This means that the original element of the argument scheme is replaced with a new one.   

\textbf{(4) Modify GRL Models} -- Based on the RationalGRL model of step (3), the GRL model can be modified: 1) a new intentional element or a new link is introduced; 2) an existing intentional element or an existing link gets disabled (removed) from the model; or 3) an existing intentional element or link is replaced by a new one. This results in a new, modified GRL model, which can be used as the basis for another cycle of the methodology. 

We can continue these four steps until there is no more intentional element or link to analyze or we reach a satisfactory model. In the next section, we will give an example of how our tool can be used together with the methodology to build a GRL model.  

\begin{figure*}[t]
\centering
\includegraphics[scale=0.8]{img/tool/goal_details}
\caption{Overview of the RationalGRL tool}
\label{fig:tool:overview}
\end{figure*}


\subsection{The RationalGRL Tool}
\label{sect:tool}

The final requirement of our framework is that it has tool support (\textbf{requirement 5}). This is important for various reasons: i) although there are various approaches attempting to combine goal modeling with argumentation, we are not aware of existing tool support (see Section~\ref{sect:discussion}), ii) it allows us to do user tests in future research, exploring the difference between GRL and RationalGRL, iii) tool support is an excellent demonstration to show that our formal framework and algorithms actually work. In this section we briefly highlight some features of the tool, and we explain some of its current limitations. The tool can be accessed from:

\begin{quote}
\url{http://www.rationalgrl.com}
\end{quote}

\paragraph{Relation to jUCMNav} 

GRL is implemented in the open-source tool jUCMNav~\cite{jUCMNav}, which is actively maintained\footnote{\url{http://jucmnav.softwareengineering.ca/foswiki/ProjetSEG}}. jUCMNav currently has over 2,000 commits made by over 40 contributors, representing over 200,000 lines of code. jUCMNav is used actively in research, and the tool has been extended in many directions over the past few years, such as for modeling and analyzing security misuse cases~\cite{daramola2012ontology}, supporting activity theory \cite{georg2015synergy}, combined goal and feature model reasoning~\cite{liu2014combined}, and enterprise architecture modeling \cite{marosin-etal:caise2016}. Furthermore, jUCMNav includes Use Case modeling and contains various GRL evaluation algorithms with which the satisfiability of goals can be determined. 

The jUCMNav tool is implemented as an Eclipse plugin. While we aim to integrate our framework into jUCMNav in the future, a web-based version of RationalGRL (and GRL) is more suited for current purposes. Since RationalGRL does not make use of the many features of jUCMNav, we believe a light-weight version is sufficient for our current aims, and may benefit the community in general. Our tool is thus not meant to replace jUCMNav. For simplicity we have a web-based version, and it has the functionality to export to jUCMNav.

\paragraph{Tool overview} The RationalGRL tool is an open-source web-based Javascript application, which runs on all modern browsers. It is based on our formalization in Section~\ref{sect:formalframework} and provides export functionality to jUCMNav, using the translation algorithm (Algorithm~\ref{alg:translation:to-grl}). It contains all of the GRL elements and links, except for beliefs and actors. We have argued in Section~\ref{sect:background:grl} that arguments can be seen as an extension to beliefs, which is the reason why we did not implement them. Actors are also missing, but they will be added in the future.

When the tool starts, the user is presented with a screen as in Figure~\ref{fig:tool:overview}. This screen shows the palette of elements and links (top left pane), a canvas on which RationalGRL models can be built (top right pane), an `Export to GRL' button (bottom left pane), and a details pane of the currently selected elements (bottom right pane). Elements and links can be added to the canvas by selecting them on the left and clicking on the canvas, thus capturing the \textsf{INTRO} operation (Section~\ref{sect:formalframework:intro}). This makes it possible to build and modify GRL models using the tool. 

In total there are four different types of details panes, which we now explain in turn.

\paragraph{IE and links details pane} Figure~\ref{fig:tool:overview} shows the details pane for the goal `Generate cars'. The user can change the name of the IE by changing the name and clicking `Rename', which will update the naming history on the bottom of the pane. The naming history is (simplified) implementation of the \textsf{REPLACE} operation on IEs, since it is a very simple form of \emph{clarification}. The main difference with jUCMNav is that we have a renaming history, so the user can see which names the IE had prior to the current one. If the IE has decomposition links, the user can change the decomposition type as well. Each IE and link has a set of associated critical questions, which can be answered in the details pane by click on the button next to the question.

\paragraph{Critical question details pane} The details panel can be used to answer the critical questions of the RationalGRL framework (Table~\ref{table:argument-schemes}): Figure~\ref{fig:tool:overview}, for example, shows two critical questions for the goal IE ``Generate cars'', namely ``Can the goal be realized?'' (CQ3) and ``Is the goal relevant/useful?'' (CQ11). As an example, consider Figure~\ref{fig:tool:cqdetails}, where the softgoal ``Easy to use'' is questioned with the relevancy question (CQ11). It is possible to select the answer and provide an explanation for the answer. 

Assume that in the example of Figure~\ref{fig:tool:cqdetails} the user selects ``No'' and clicks the ``Answer question'' button. A new argument is then automatically created that attacks the softgoal and the details pane shows the critical question as answered (Figure~\ref{fig:tool:cqeffect}). This is an implementation of a \textsf{DISABLE} algorithm (Section~\ref{sect:formalframework:disable}) similar to Algorithm~\ref{alg:actor-not-relevant}: a new argument is added that attacks the existing argument. 

\paragraph{Argument details pane} It is also possible to attack arguments by adding an ``Argument'' element and an ``Attack'' link manually. Consider, for example, Figure~\ref{fig:tool:argument}. Here, a new argument ``Necessary'', which attacks the previously generated argument based on CQ11, has been added by the user. As the details pane shows, this new argument is not based on a CQ. It is further worth noting that it is possible to provide further a explanation in argument elements, allowing for more fine-grained rationalizations. 

\begin{figure}[t]
\centering
\includegraphics[width=0.8\columnwidth]{img/tool/details_softgoal}
\caption{Critical question details pane for ``Is the softgoal relevant/useful?''}
\label{fig:tool:cqdetails}
\end{figure}

\begin{figure}[b]
\centering
\includegraphics[width=\columnwidth]{img/tool/attack_softgoal}
\caption{Effect of answering CQ11: ``Is the softgoal relevant/useful?'' with ``No''}
\label{fig:tool:cqeffect}
\end{figure}

\paragraph{Argumentation semantics} The RationalGRL tool computes the acceptability of arguments on the fly. In the example of Figure~\ref{fig:tool:cqeffect}, the original (argument for) softgoal ``Easy to use'' is rejected because its only attacker ``CQ11 - Redundant'' is accepted. However, if we then attack ``CQ11 - Redundant'' with a new argument ``Necessary'', the original argument for ``Easy to use'' is again accepted because its only attacker is rejected  (Figure~\ref{fig:tool:argument}). Note that when computing the acceptability of arguments, the RationalGRL tool makes the assumption that there are no attack cycles in the model and that hence there is one unique preferred extension (cf. Section~\ref{sect:formalframework:rationalgrl}) -- if the user creates an attack cycle, an error message is shown. 

\begin{figure}[t]
\centering
\includegraphics[width=0.8\columnwidth]{img/tool/reinstate_softgoal}
\caption{Attacking an argument with a generic argument ``Necessary''}
\label{fig:tool:argument}
\end{figure}

\paragraph{Argument schemes and critical questions} The tool contains more argument schemes and critical questions than those that we initial collected in Table~\ref{table:argument-schemes}. As we mentioned before, our table is not meant to be exhaustive, and it is straightforward to add more argument schemes and critical questions if necessary. For instance, the table does not contain critical questions for links between all IEs (e.g., a contribution link from a task to a goal does not occur in the table). In the tool, we have added critical questions to all links and IEs. We implemented an \emph{argument schemes database}, which is designed to be extended easily.


\paragraph{Export to jUCMNav} The tool implements the RationalGRL to GRL translation (Algorithm~\ref{alg:translation:to-grl}) through an export function. RationalGRL models built in the RationalGRL tool can be exported to the \texttt{.grl} file format, which can be imported in jUCMNav. Note that in order for the GRL models to be rendered correctly, we recommend using the Graphviz dot\footnote{\url{http://www.graphviz.org/}} as the auto-layout tool (this can be selected under ``Auto layout preferences'' when importing the file in jUCMNav).

Two examples of this export are provided in Appendix~\ref{sect:tool-screenshots}. Figure~\ref{fig:tool:figfrompaper} shows the model that was discussed earlier in this paper (Figure~\ref{fig:example-small3}) in the RationalGRL tool. Recall that translating this model to GRL provided us with the model in Figure~\ref{fig:example-small}. This is also what follows from our export: if we export the RationalGRL model and then import the resulting GRL model in jUCMNav, we get the model in Figure~\ref{fig:tool:figfrompaper1}, which is the same as Figure~\ref{fig:example-small}.

Our translation and export function uses the argument acceptability as a way of determining the GRL model. Take for example, the RationalGRL model in Figure~\ref{fig:tool:multipleattack}. A positive contribution is added from ``Keep same cars'' to ``Simple design''. More inportantly, an argument ``not enough CPU'' has been added to the model. This argument attacks the ``Realistic simulation'' softgoal and the ``Create new cars'' task, arguing that there is not enough processing power for either of these to be feasible in the traffic simulator. Now, if we export this to GRL, the GRL elements that are rejected or disabled (greyed out in Figure~\ref{fig:tool:multipleattack}) are not included. This can be seen in Figure~\ref{fig:tool:multipleattack1}, where the jUCMNav GRL model that was exported from the RationalGRL tool is shown. The pair of figures \ref{fig:tool:multipleattack} and \ref{fig:tool:multipleattack1} also nicely shows the added value of RationalGRL: Figure~\ref{fig:tool:multipleattack} shows that there can be a larger discussion and rationalization underlying even a fairly simple goal model such as the one in Figure~\ref{fig:tool:multipleattack1}. 

\paragraph{Limitations}
For usability purposes, the \textsf{REPLACE} operation has been implemented differently than in our formal framework (Section~\ref{sect:formalframework:disable}). In the formal framework, a \textsf{REPLACE} introduces a new argument that replaces (and therefore attacks) all previous arguments with the same identifier. Including \textsf{REPLACE} like this in the tool would mean that, when the name of an element is changed, we would need to render all the previous versions of the element on the canvas of the tool. Since this would quickly become very cumbersome, we decided to implement a naming history for each element. For example, in Figure~\ref{fig:tool:overview}, the goal ``Generate cars'' was previously named ``Add cars''. In this way, a history is kept of how an element name was clarified without cluttering the model canvas. This feature is currently very limited and can be extended in various was, for instance by changing the IE type. Similarly, the decomposition type can be changed in the details pane (Figure~\ref{fig:tool:overview}) without introducing new elements that attack the original ones (cf. Figure~\ref{fig:example-small3}, where the old decomposition type $AND$ of ``Generate traffic'' is shown as an explicit argument).
\section{Related Work}
\label{sect:relatedwork}

There are several contributions that relate argumentation-based techniques with goal modeling. The contribution most closely related to ours is the work by Jureta \emph{et al.}~\cite{Jureta:RE2008}. This work proposes ``Goal Argumentation Method (GAM)'' to guide argumentation and justification of modeling choices during the construction of goal models. One of the elements of GAM is the translation of formal argument models to goal models (similar to ours). In this sense, our \textsf{RationalGRL} framework can be seen as an instantiation and implementation of  part of the GAM. One of the main contribution of \textsf{RationalGRL} is that it also takes the acceptability of arguments as determined by the argumentation semantics \cite{Dung1995} into account when translating from arguments to goal models.  \textsf{RationalGRL} also provides tool support for argumentation, i.e. Argument Web toolset, to which OVA belongs \cite{bex2013implementing}, and for goal modeling, i.e. jUCMNav~\cite{jUCMNav}. Finally, \textsf{RationalGRL} is based on the practical reasoning approach of \cite{Atkinson2014}, which itself is also a specialization of Dung's~\cite{Dung1995} abstract approach to argumentation. Thus, the specific critical questions and counterarguments based on these critical question proposed by~\cite{Atkinson2014} can easily be incorporated into \textsf{RationalGRL}. 

\textsf{RationalGRL} framework is also closely related to frameworks that aim to provide a design rationale (DR)~\cite{shum2006hypermedia}, an explicit documentation of the reasons behind decisions made when designing a system or artefact. DR looks at issues, options and arguments for and against the various options in the design of, for example, a software system, and provides direct tool support for building and analyzing DR graphs. One of the main improvements of \textsf{RationalGRL} over DR approaches is that \textsf{RationalGRL} incorporates the formal semantics for both argument acceptability and goal satisfiability, which allow for a partly automated evaluation of goals and the rationales for these goals. 

Arguments and requirements engineering approaches have been combined by, among others, Haley \emph{et al.}~\cite{haley2005arguing}, who use structured arguments to capture and validate the rationales for security requirements. However, they do not use goal models, and thus, there is no explicit trace from arguments to goals and tasks. Furthermore, like~\cite{Jureta:RE2008}, the argumentative part of their work does not include formal semantics for determining the acceptability of arguments, and the proposed frameworks are not actually implemented. Murukannaiah \emph{et al.}~\cite{murukannaiah2015resolving} propose Arg-ACH, an approach to capture inconsistencies between stakeholders' beliefs and goals, and resolve goal conflicts using argumentation techniques.
\section{Conclusions and Future Work}
\label{sect:conclusion}

In this paper, we developed and implemented a framework to trace back elements of GRL models to arguments and evidence that derived from the discussions between stakeholders. We created a mapping algorithm from a formal argumentation theory to a goal model, which allows us to compute the evaluation values of the GRL IEs based on the formal semantics of the argumentation theory. 

There are many directions of future work. There are a large number of different semantics for formal argumentation, that lead to different arguments being acceptable or not. It would be very interesting to explore the effect of these semantics on goal models. Jureta \emph{et al.} develop a methodology for clarification to address issues such as ambiguity, overgenerality, synonymy, and vagueness in arguments. Atkinson \emph{et al.}~\cite{atkinson2007} define a formal set of critical questions that point to typical ways in which a practical argument can be criticized. We believe that critical questions are the right way to implement Jureta's methodology, and our framework would benefit from it. In addition, currently, we have not considered the \emph{Update} step of our framework (Figure~\ref{fig:overview}). That is, the translation from goal models to argument diagrams is still missing. The \emph{Update} step helps analysts change parts of the goal model and analyze its impact  on the underlying argument diagram. Finally, the implementation is currently a browser-based mapping from an existing argument diagramming tool to an existing goal modeling tool. By adding an argumentation component to jUCMNav, the development of goal models can be improved significantly. 

\section*{Acknowledgments}
Marc van Zee is funded by the National Research Fund (FNR), Luxembourg, by the Rational Architecture project.

\bibliographystyle{abbrv}
\bibliography{references}

\newpage

\onecolumn
\appendix

\section{Transcripts excerpts}
\label{sect:transcripts:excerpts}

\begin{table}[!htbp]
\centering
\begin{tabular}{|p{20mm}|p{70mm}|p{60mm}|}
\hline
Respondent & Text & Annotation\\
\hline
0:10:55.2 (P1) & Maybe developers &\textbf{[4 actor (AS0)]} Development team\\
\hline
0:11:00.8 (P2) & Development team, I don't know. Because that's- in this context it looks like she's gonna make the software & \textbf{[5 critical question CQ0 for 4]} Is actor "development team" relevant?\newline
\textbf{[6 answer to 5]} No, it looks like the professor will develop the software.\\
\hline
\end{tabular}
\caption{AS0: Actor, CQ0: Relevant actor? (transcript $t_3$)}
\label{table:transcript:irrelevant-actor}

\begin{tabular}{|p{20mm}|p{70mm}|p{60mm}|}
\hline
Respondent & Text & Annotation\\
\hline
0:19:08.6 (P3) & Should have a link with an outsource program for the statistical distribution [inaudible] & \textbf{[21 resource (AS1)]} Actor System has resource ``Statistics library''\\
\hline
0:35:27.4 (P3) & Maybe before traffic simulation view you can- the outsource package that makes the map	& \textbf{[38 contribution (AS8)]} Resource ``Statistics library'' contributes to task ``Display traffic simulation''\\
\hline
\end{tabular}
\caption{AS1: Resource, AS8: Resource contributes to task (transcript $t_3$)}
\label{table:transcript:as1-as8}

\begin{tabular}{|p{20mm}|p{70mm}|p{60mm}|}
\hline
Respondent & Text & Annotation\\
\hline
0:15:11.2 (P1) & And then, we have a set of actions. Save map, open map, add and remove intersection, roads & \multirow{2}{60mm}{\textbf{[20 task (AS2)]} Student has tasks ``save map'', ``open map'', ``add intersection'', ``add road'', ``add traffic light'', ``remove intersection''}\\
\cline{1-2}
0:15:34.7 (P2) & Yeah, road. Intersection, add traffic lights	&\\
\hline
0:15:42.3 (P1) & Well, all intersection should have traffic lights so it's & \textbf{[21 critical question CQ*1 for 20]} Is the task ``Add traffic light'' useful/redundant? \newline
\textbf{[22 answer to 22]} Not useful, because according to the specification all intersections have traffic lights.\\
\hline
0:15:52.3 (P1) & (..) And on the technical side it's gonna be a real pain to remove one intersection you're gonna have to remove a lot more because there are only four-ways allowed and if you remove one intersection then-& \textbf{[23 critical question CQ2 for 20]} Is the task ``Remove intersection'' possible?\newline
\textbf{[24 answer to 22]} It is going to be very difficult to implement.\\
\hline
\end{tabular}
\caption{AS2: Task, CQ*1: Redundant element, CQ2: impossible task (transcript $t_1$)}
\label{table:transcript:as2-cq_star_1-cq2}

\begin{tabular}{|p{20mm}|p{70mm}|p{60mm}|}
\hline
Respondent & Text & Annotation\\
\hline
0:23:20.4 (P1) & So, sets, yeah, car influx & \textbf{[32 task (AS2)]} Student has task ``car influx''\\
\hline
0:23:41.2 (P2) & (..) If you can only control the set amount of influx from any side of this sort of random distribution, I think that is going to be less interesting than when you can say something like, this road is frequently traveled. (...) So setting it per road, I think is something we want & \textbf{[33 critical question CQ*2 on 36]} Is the task description specific/clear enough? \newline
\textbf{[34 answer to 37]} No, it is not clear where the influx is changing. Change to ``control car influx per road''\\
\hline
\end{tabular}
\caption{AS2: Task, CQ*2: Specify/clarify element (transcript $t_1$)}
\label{table:transcript:as2-cq_star_2}

\begin{tabular}{|p{20mm}|p{50mm}|p{80mm}|}
\hline
Respondent & Text & Annotation\\
\hline
0:14:31.2 (P1) & Let's see- she uses the system in her course to explain her lectures about traffic problem thing & \textbf{[11 softgoal (AS4)]} ``Explain lectures traffic theory'' (Professor)\newline
\textbf{[12 goal (AS5)]} ``Use traffic light system'' (Professor)\newline
\textbf{[13 contribution (AS7)]} ``use traffic light system'' contributes to ``explain lectures traffic theory''\\
\hline
\end{tabular}
\caption{AS4: softgoal, AS5: goal, AS7: contribution (transcript $t_3$)}
\label{table:transcript:as4-as5-as7}

\begin{tabular}{|p{20mm}|p{90mm}|p{40mm}|}
\hline
Respondent & Text & Annotation\\
\hline
0:00:10.2\newline PERSON 1 & 	So, yeah [pause] I would start with something about the context. That we have to determine who the users of the system are gonna be, stakeholders. & \textbf{[1 topic]} What are the actors?\\
\hline
\end{tabular}
\caption{AS*: Topic introduction (transcript $t_1$)}
\label{table:transcript:as-star}
\end{table}

\begin{table}[!htbp]
\centering
\begin{tabular}{|p{20mm}|p{70mm}|p{60mm}|}
\hline
Respondent & Text & Annotation\\
\hline
0:29:59.5 (P3) & (...) a variety of sequences and timing schemes should be allowed.  (...) we would have traffic light behavior gives you, I guess two options then. & \multirow{4}{60mm}{\textbf{[42 task (AS2)]} Student has task ``set sequence scheme''\newline
\textbf{[43 task (AS2)]} Student has task ``set timing scheme'' \newline
\textbf{[44 decomposition (AS10)] }Task ``set traffic light behavior'' XOR-decomposes into ``set sequence scheme'' and ''set timing scheme''}\\
\cline{1-2}
0:30:23.6 (P1) & Sequences and timing schemes &\\
\cline{1-2}
0:30:25.0 (P3) & Sequences and timing schemes. So you can either go for, yeah, sequences-&\\
\cline{1-2}
0:30:30.9 (P1) & Or timing schemes&\\
\hline	
\end{tabular}
\caption{AS2: Task, AS10: Task decomposition (transcript $t_2$)}
\label{table:transcript:as2-as10}

\begin{tabular}{|p{20mm}|p{70mm}|p{60mm}|}
\hline
Respondent & Text & Annotation\\
\hline
0:30:10.3 (P1) & 	Yeah. But this is- is this an OR or an AND? & \multirow{3}{60mm}{\textbf{[26 critical question CQ*0 for 20]} Is the OR-decomposition of ``simulate'' correct?\newline
\textbf{[27 answer to 26]} No, it should be an AND because the system can do both.}\\
\cline{1-2}
0:30:14.3 (P3) &I think it's an OR&\\
\cline{1-2}
0:30:15.4 (P1) & for the data, it's an OR (...) And for the system it's an AND. (...) Static manner or dynamic. But the system can do both&\\
\hline	
\end{tabular}
\caption{CQ11b: Incorrect decomposition (transcript $t_3$)}
\label{table:transcript:cq11b}

\begin{tabular}{|p{20mm}|p{100mm}|p{30mm}|}
\hline
Respondent & Text & Annotation\\
\hline
0:06:29.3 (P2) & So, is that a trade-off. I think so. &\multirow{2}{30mm}{\textbf{[10 negative contribution (AS11)]}  task ``generate cars randomly'' contributes negatively to softgoal ``dynamic simulation''}\\	
\cline{1-2}
0:06:36.0 (P1) & Yeah, performance versus, I don't know, functionality. Like, what you say, cars come out at the end of the map side [are generated randomly] is performance wise and, I don't know, easier to make but it is less functional. Because you can't see traffic flows that easy because, well there's fixed amount of cars so there's not really gonna be jams [the simulation is less dynamic]. Is there around Utrecht always the same amount of cars? &\\
\hline	
\end{tabular}
\caption{AS11: Negative decomposition (transcript $t_1$)}
\label{table:transcript:as11}

\begin{tabular}{|p{20mm}|p{80mm}|p{50mm}|}
\hline
Respondent & Text & Annotation\\
\hline
0:49:05.3 (P3)&So, density, speed and, is there anything else.&\multirow{3}{50mm}{\textbf{[68 critical question for 63c]} Does ``set road characteristics'' decompose into any other tasks?\newline
\textbf{[69 answer to 68]} Yes, type of cars.}\\
\cline{1-2}
0:49:20.1 (P1) & Maybe type of cars&\\	
\cline{1-2}
0:49:22.0 (P3) & Type of cars, because you could have trucks, you could have personal cars. That would be good because-&\\
\hline	
\end{tabular}
\caption{CQ: Does a task decompose into other tasks? (transcript $t_2$)}
\label{table:transcript:cq:task_decomp}

\begin{tabular}{|p{20mm}|p{60mm}|p{70mm}|}
\hline
Respondent & Text & Annotation\\
\hline

0:10:55.2 (P1) & Maybe developers or & \textbf{[4 actor (AS0)]} Development team\\
\hline
0:11:00.8 (P2)&Development team, I don't know. Because that's- in this context it looks like she's gonna make the software&\textbf{[5 critical question CQ0 for 4]} Is actor ``development team'' relevant?\newline
\textbf{[6 answer to 5]} No, it looks like the professor will develop the software.\\
\hline
..&..&..\\
\hline
0:18:13.4 (P2) & I think we can still do developers here. To the system & \multirow{2}{70mm}{\textbf{[16 counter argument for 6]} According to the specification the professor doesn't actually develop the software.}\\
\cline{1-2}
0:18:22.9 (P1)&Yeah, when the system gets stuck they also have to be [inaudible] ok. So development team&\\	
\hline	
\end{tabular}
\caption{AS0: Task, CQ0: Relevant task? CQ: Generic counterargument (transcript $t_2$)}
\label{table:transcript:as0-cq0-cq_counterarg}

\end{table}


\end{document}