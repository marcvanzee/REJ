\section{Introduction}
\label{sect:introduction}

\subsection{Requirements Engineering and Goal Modeling}

Requirements Engineering (RE) is an approach to assess the role of a future information system within a human or automated environment. An important goal in RE is to produce a consistent and comprehensive set of requirements covering different aspects of the system, such as general functional requirements, operational environment constraints, and so-called non-functional requirements such as security and performance. 

One of the first activities in RE are the ``early-phase'' requirements engineering activities, which include those that consider how the intended system should meet organizational goals, why it is needed, what alternatives may exist, what implications of the alternatives are for different stakeholders, and how the interests and concerns of stakeholders might be addressed~\cite{yu1997towards}. These activities fall under the umbrella of goal modeling. The are a large number of established RE methods using goal models in the early stage of requirements analysis (e.g.,~\cite{liu2004designing,donzelli2004goal,dardenne1993goal,chung2012non,castro2002towards}, overviews can be found in~\cite{van2001goal,kavakliL05}). Several goal modeling languages have been developed in the last two decades as well. The most popular ones include $i*$~\cite{Yu:1997:TMR:827255.827807}, Keep All Objects Satisfied (KAOS)~\cite{van2008requirements}, the NFR framework~\cite{chung2012non}, \textsc{Tropos}~\cite{giorgini2005goal}, the Business Intelligence Model (BIM)~\cite{horkoff2014strategic}, and the Goal-oriented Requirements Language (GRL)~\cite{Amyot:2010:EGM:1841349.1841356}.

\subsection{Problem Statement}
\label{sect:problem_statement}


A goal model is often the result of a discussion process between a group of stakeholders. For small-sized systems, goal models are usually constructed in a short amount of time, involving stakeholders with a similar background. Therefore, it is often not necessary to record all of the details of the discussion process that led to the final goal model. However, most real-world information systems -- e.g., air-traffic management, industrial production processes, or government and healthcare services -- are complex and are not constructed in a short amount of time, but rather over the course of several workshops. In such situations, failing to record the discussions underlying a goal model in a structured manner may harm the success of the RE phase of system development for various reasons stated below. 

First, it is well-known that stakeholders' preferences are rarely absolute, relevant, stable, or consistent~\cite{march1978bounded}. Therefore, it is possible that a stakeholder changes his or her opinion about a modeling decision in between two goal modeling sessions, which may require revisions of the goal model. If previous preferences and opinions are not stored explicitly, it is not straightforward to remind stakeholders of their previous opinions which results in having unnecessary discussions and revisions. As the number of participants increases, revising the goal model based on changing preferences can take up a significant amount of time. 

Second, other stakeholders, such as new developers on the team who were not the original authors of the goal model, may have to make sense of the goal model, for instance, to use it as an input in a later RE stage or at the design and development phase. If these users have no knowledge of the underlying rationale of the goal model, it may not only be more difficult to understand the model, but they may also end up having the same discussions as the previous group of stakeholders.

Third, alternative and different ideas and opposing views that could potentially have led to different goal diagrams could be lost. For instance, a group of stakeholders specifying a goal model for a user interface may decide to reduce softgoals ``easy to use'' and ``fast'' to one softgoal ``easy to use''. Thus, the resulting goal model merely contains the softgoal ``easy to use'', but the discussion as well as the decision to reject ``fast'' are lost. This leads to a poor understanding of the problem and solution domain. In fact, empirical data suggest that this is an important reason of RE project failure~\cite{curtis1988field}. 

Fourth and lastly, in goal models in general, ``rationale'' behind any decision is static and does not immediately impact the goal models when they change. That is, current goal modeling languages have limited support for reasoning about changing beliefs and opinions, and their effect on the goal model. A stakeholder may change his or her opinion, but it is not always directly clear what its effect is on the goal model. Similarly, a part of the goal model may change, but it is not possible to formally reason about the consistency of this new goal model with the underlying beliefs and arguments in the goal modeling languages. This becomes more problematic if the participants constructing the goal model change, since modeling decisions made by one group of stakeholders may conflict with the underlying beliefs put forward by another group of stakeholders.

\subsection{Research Questions}
 
The aim of this research is to resolve the above issues by developing a framework with tool-support, which combines goal modeling approaches with argumentation techniques from Artificial Intelligence (AI) research~\cite{atkinson2007}. We have identified five important requirements for our framework: 
\begin{enumerate}
\item The argumentation techniques should be close to the actual discussions of stakeholders or designers in the early requirements engineering phase.
\item 
The framework must have formal traceability links between elements of the goal model and underlying arguments.
\item 
Using these traceability links, it must be possible to compute the effect of changes in the goal model on the underlying arguments, and vice versa.
\item 
The framework must have a tool support.
\item 
There should be a methodology for the framework to guide the practitioners in its application in real cases.
\end{enumerate}

The five requirements above serve as the success criteria of our approach. That is, if our framework satisfies these requirements, we reach our goal. 

In this context, the main research question is: 

\begin{quote}
\textbf{RQ.} What are the constructs, mechanisms and rules for developing a framework that formally captures the discussions between stakeholders such that it can generate goal models?
\end{quote}

To answer the research question above and solve some of the concerns mentioned in Subsection~\ref{sect:problem_statement}, we propose a framework called RationalGRL. The RationalGRL framework combines the Goal-oriented Requirements Language (GRL) with a technique from argumentation and discourse modeling called \emph{argument schemes}~\cite{walton-etal2004}. 

In past, we introduced our preliminary steps towards RationalGRL framework~\cite{bnaic2014,vanzee-etal:renext2015,vanZee-etal:er2016} with its tool-support~\cite{vanZee-etal:comma2016}.  In our previous work, argument diagrams are integrated with GRL models to allow better representation of stakeholder's arguments and discussions. In our current paper, we improve RationalGRL framework by adding argument schemes and critical questions to argumentation framework and provide traceability links between the argumentation framework and GRL.  For this, we modify and extend our metamodel to specify how our framework extends GRL, and we use this as the basis for a prototype web-based implementation~\footnote{The tool is available at <GITHUB LINK>}. %SG: Placeholder for the link. 
We develop an initial list of argument schemes and critical questions by analyzing a set of transcripts containing discussion about an information system. Our framework is fully extensible, meaning that argument schemes and critical questions can be added by users for specific problem domains. We develop a methodology for using RationalGRL, which consists of developing goal models and posing arguments in an integrated way. This methodology is intended to be used by practitioners in the field. We formalize RationalGRL using propositional logic, and develop algorithms for adding goal elements and arguments. We illustrates the steps of our framework  and how the framework can be applied in real cases with a traffic simulator example.

\subsection{Organization} %SG: Will get back to this at the end of the paper.

This article is organized as follows. Section~\ref{sect:background} contains background and introduces our running example, the Goal-oriented Requirements Language (GRL)~\cite{Amyot:2010:EGM:1841349.1841356}, and argument schemes. Section~\ref{sect:overview} provides a brief and high-level overview of our framework, together with a metamodel and the methodology. Section~\ref{sect:gmas} contains an in depth explanation of how we obtained an initial set of argument schemes by annotating transcripts from discussions about an information system, and in Section~\ref{sect:examples} we provide several examples of these arguments schemes. In Section~\ref{sect:formalframework} we formalize RationalGRL and our arguments schemes using propositional logic and we use well-known argumentation semantics to compute which arguments are accepted and which are rejected. We also develop various algorithms for the argument schemes in this section. In Section~\ref{sect:tool} we provide a brief overview of the prototype tool we developed for RationalGRL. Finally, Section~\ref{sect:discussion} contains a discussion, covering related work, future work, and a conclusion.