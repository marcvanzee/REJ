\section{Argument Schemes for Goal Modeling: a Case Study}
\label{sect:gmas}


The objective of our case study was to see which types of discourse are used during a discussion of system requirements, and how these discourse types can be captured as argument schemes and critical questions regarding goals and tasks. In order to study this, we manually coded transcripts of such discussions. For our coding, we used a list of argument schemes and critical questions based on GRL and PRAS. In this section we present our case study. All original transcripts, codings, and models are available in our online repository\footnote{\label{foot:url}\url{insertURL}}.   

In order to obtain actual requirements discussions, we turned to a recent series of experiments by Schriek et al. \cite{SchriekEtal2016}. In these experiments, Schriek and colleagues gave the traffic simulator assignment (Appendix~\ref{sect:designprompt}) to 12 groups of two or three students in a Software Architecture course at MSc level. These groups had a maximum of two hours to design a traffic simulator, which included a discussion of the requirements of this traffic simulator. The students did not use any goal modeling technique in the course or during the discussions. The students were asked to record their design sessions, and the recordings were subsequently transcribed. We used three of these transcripts, totaling 153 pages, for our case study. 

Before we started coding the transcripts, we drew up an initial list of 10 argument schemes (AS0-AS9 in Table~\ref{table:transcripts:results:argumentschemes}), representing \emph{claims} about the goal model containing the requirements of the system. AS0 to AS4 are schemes that concern a single element of the goal model. So, for example, AS0 represents the claim `$a$ is a relevant actor for the system', and AS3 represents the claim `$G$ is a goal for the system'. AS5 to AS9 are claims about the links between GRL elements. Our initial list also contained 18 critical questions, inspired by the questions associated with the original Practical Reasoning Argumentation Scheme \cite{atkinson2007}. CQ2 to CQ6b (Table~\ref{table:transcripts:results:argumentschemes}) are examples of these critical questions, other examples are `Is the softgoal legitimate?' and `Are there alternative ways to contribute to the same softgoal?'.

\begin{table}[t]
\centering
\begin{tabularx}{0.5\textwidth}{|l|X|l|l|l|>{\bfseries}l|}
\hline
\multicolumn{2}{|c|}{\textbf{Scheme/Question}} & $t_1$ & $t_2$ & $t_3$ & \textbf{total}\\
\hline 
AS0 & Actor & 2 & 2 & 5 & 9\\
\hline
AS1 & Resource & 2 & 4 & 5 & 11\\
\hline
AS2 & Task/action & 20 & 21 & 17 & 58\\
\hline
AS3 & Goal & 0 & 2 & 2 & 4\\
\hline
AS4 & Softgoal & 3 & 4 & 2 & 9\\
\hline
AS5 & Goal decomposes into tasks & 4 &0& 4 & 8\\
\hline
AS6 & Task contributes (negatively) to softgoal & 8 & 3 &0& 11\\
\hline
AS7 & Goal contributes (negatively) to softgoal &0& 1 & 1 & 2\\
\hline
AS8 & Resource contributes to task & 0 & 4 & 3 & 7\\
\hline
AS9 & Actor depends on actor &0& 1 & 3 & 4\\
\hline
AS10 & Task decomposes into tasks & 11 &14 &11 &36\\ 
\hline
\hline
CQ2 & Task is possible? & 2 & 2 & 1 & 5\\
\hline		
CQ5a & Does the goal decompose into the tasks? & 0 & 1 & 0 & 1\\
\hline
CQ5b & Goal decomposes into other tasks? & 1 & 0 & 0 & 1\\
\hline
CQ6b & Task has negative side effects? & 2 & 0 & 0 & 2\\
\hline
CQ10a & Task decompose into other tasks? & 1 &2 &0&3\\
\hline
CQ10b & Decomposition type correct? &1 &0& 1 &2\\
\hline
\hline
CQ12 & Is the element relevant/useful? & 2 & 3 & 2 &7\\
\hline
CQ13 & Is the element clear/unambiguous? &3 &10 & 3 & 16\\
\hline
\hline
- & Generic counterargument & 0& 2 & 2 & 4\\
\hline
\hline
\multicolumn{2}{|c|}{\textbf{TOTAL}}&69&80&69&222\\
\hline
\end{tabularx}
\caption{Occurrences of argument schemes and critical questions in the transcripts.\todo{F}{F}{check numbers against original annotation docs}}
\label{table:transcripts:results:argumentschemes}
\end{table}

Using the initial list of arguments and critical questions, we coded three transcripts of requirements discussions. The codings were performed by one author and subsequently checked by another author. As the transcripts contain spoken language, the codings involved some interpretation. For example, the students almost never literally say `actor $a$ has task $T$'. Rather, they say things such as `...we have a set of actions. Save map, open map, ...' (Table~\ref{table:transcripts:traffic-light}, Appendix~\ref{sect:transcripts:excerpts}) and `... in that process there are activities like create a visual map, create a road' (Table~\ref{table:transcript:task-clarification}, Appendix~\ref{sect:transcripts:excerpts}). Furthermore, in some cases the critical questions are explicit. For example, CQ10b is found in the transcripts as `...is this an OR or an AND?' (Table \ref{table:transcript:decomposition}, Appendix~\ref{sect:transcripts:excerpts}). In other cases, however, the question remains implicit but we added it in the coding. For example, CQ12 is not found in the transcripts, but the related counterargument is:  `...you don't have to specifically add a traffic light' (Table~\ref{table:transcripts:traffic-light}, Appendix~\ref{sect:transcripts:excerpts}). 

In order to further verify the coherence of our codings, we constructed complete RationalGRL models based on the argument schemes and critical questions found in each transcript. An example model is shown in Figure~\ref{fig:transcripts:grl}, and the full models can be found in the repository. More examples of RationalGRL models based on our codings will be given in section \ref{sect:overview}.

During the coding, new argument schemes and critical questions were added to the list. For example, we found that the discussants often talk about tasks decomposing into sub-tasks, so we added AS10 and CQ10a. Furthermore, because there were many discussions on the relevance and the clarity of the names of elements (goals, tasks, etc.), two generic critical questions CQ12 and CQ13 were added. The final results of the coding can be found in Table~\ref{table:transcripts:results:argumentschemes}; for results per transcript, please consult the online repository. We found a total of 159 instantiation of argument schemes AS0-AS11 in transcripts. The most used argument scheme was AS2: ``Actor $A$ has task $T$'', however, each argument scheme is found in transcripts at least twice. A large portion (about 60\%) of argument schemes involved discussions about tasks of the information system (AS2, AS10). We coded 41 applications of critical questions. Many critical questions (about 55\%) involved clarifying the name of an element, or discussing its relevance (CQ12, CQ13).


%MvZ: I completely removed the subsection below. I think it doesn't add much and it is confusing. What do you think?  

%SG: I agree... It was more confusing and downgrade our approach

%\subsection{Analysis}
%\label{sect:gmas:transcripts:analysis}

%\paragraph{Analysis of Argument Schemes}
%Recall that our initial list of argument schemes consists of AS1-AS4, AS6-AS9 (Table~\ref{table:argument-schemes}). Therefore, the difference between the initial list of argument schemes and those found in transcripts is quite small. 
%TODO The next few sentences are weird... they make our work very trivial.. rewrite.
%We found it surprising that we were able to find back all the schemes in the transcript at least twice, even more since the topic of discussion was not goal models, but more generally the architecture of an information system. This gives us an indication that it is possible to capture (parts of the) arguments used in those type of discussions using argument schemes.

%We observed that our initial list is rather limited, which is a consequence of the fact that it is derived from PRAS. Since PRAS only considers very specific types of relationships, we are not able to capture many other relationships existing in GRL. GRL has four types of intentional elements (softgoal, goal, task, resource) and four types of relationships (positive contribution, negative contribution\footnote{In fact, a contribution can be any integer in the domain [-100,100], but for the sake of simplicity we only consider two kinds of contributions here.}, dependency, decomposition), allowing theoretically $4^3=64$ different types of argument schemes, of which we currently only consider 11. 
%TODO which analysis? why aren't they used? How did we empirically validate it? How are we confident in the result?
%Our analysis, however, shows that many of these schemes are not often used, and thus, gives us some confidence in the resulting list. However, additional argument schemes and critical questions can be easily added to our framework. Beside, our list is not meant to be exhaustive.

%\paragraph{Analysis of the Critical Questions} The difference between the initial list of critical questions and those we found in transcripts is much larger than for the critical questions. %TODO  I don't get this sentence. Also, the ones below. 
%We found few of the critical questions we initially proposed. %TODO ???
%However, this does not mean that they were not implicitly used in the minds of the participants. If a participant, for instance, forms an argument for a contribution from a task to a softgoal, it may very well be that she/he was asking her/him-self the question ``Does the task contributes to some other softgoals?''. However, many of these critical questions are not mentioned explicitly. If we assume this explanation is at least partially correct, then this would mean that critical questions may still play a role when formalizing the discussions leading up to a goal model, and it would be limiting to leave them out of our framework. In the context of a tool support, we believe that having these critical questions available may stimulate discussions. %TODO what does it mean in the context of a tool support?