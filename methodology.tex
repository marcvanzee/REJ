\section{The RationalGRL Methodology and Tool}
\label{sect:methodology}

In previous sections, we have shown how the RationalGRL framework can capture stakeholder discussions, and how interactions between two types of reasoning, practical reasoning and goal modeling, leads to two interlinked models, RationalGRL and GRL models. In this section we clarify how practitioners can actually use the RationalGRL framework by proposing a methodology (\textbf{requirement 4}) and discussing a prototype RationalGRL tool (\textbf{requirement 5}).

\subsection{RationalGRL Methodology}
\label{sect:methodology} 

We propose the methodology shown in Figure~\ref{fig:rationalgrl-methodology} to develop a (Rational)GRL model, an earlier version of which was presented at the 2017 iStar workshop \cite{ghanavatiMethodology}. %Here we assume that the initial GRL models have been created based on the requirements specification documents and the discussions of the stakeholders. The rest of the steps are as follows:

\textbf{(1) Instantiate Argument Schemes (AS)} -- We start with the list of argument schemes (Table~\ref{table:argument-schemes}). Whilst discussing the requirements, we select schemes from the list and instantiate them to form arguments for GRL model elements. In this way we build or modify the GRL model by introducing new GRL elements (\textsf{INTRO}). Note that it is also possible to start modifying an existing GRL model which was not built using the RationalGRL methodology: each GRL element corresponds to an argument (i.e. an instantiated scheme), so it is possible to instantiate argument schemes based on an existing GRL model. 

\textbf{(2) Answer Critical Questions (CQs)} -- After building or modifying the initial GRL model, we ask the relevant critical questions. Because each element in the GRL model corresponds to an instantiated scheme, we can look at Table~\ref{table:argument-schemes}) to see which questions are relevant given our GRL model. 

\textbf{(3) Decide on Intentional Elements and their Relationships} -- By answering a critical question, one of the \textsf{INTRO}, \textsf{DISABLE} or \textsf{REPLACE} are performed on the GRL model. Any of these operations impact the arguments and corresponding GRL intentional elements, modifying the initial GRL model into a RationalGRL model. After these modifications, we can keep on asking critical questions (e.g. about elements that were introduced by previously answering a critical question) until we are satisfied with our model.   

\textbf{(4) Modify GRL Models} -- Based on the RationalGRL model of step (3), the GRL model can be modified: 1) a new intentional element or a new link is introduced; 2) an existing intentional element or an existing link gets disabled (removed) from the model; or 3) an existing intentional element or link is replaced by a new one. This results in a new, modified GRL model, which can be used as the basis for another cycle of the methodology. 

We can continue these four steps until there is no more intentional element or link to analyze or we reach a satisfactory model. In the next section, we will give an example of how our tool can be used together with the methodology to build a GRL model.  

\begin{figure*}[t]
\centering
\includegraphics[scale=0.8]{img/tool/goal_details}
\caption{Overview of the RationalGRL tool}
\label{fig:tool:overview}
\end{figure*}


\subsection{The RationalGRL Tool}
\label{sect:tool}

An important requirement of our framework is that it has tool support (\textbf{requirement 5}. This is for various reasons: i) although there are various approaches attempting to combine goal modeling with argumentation, there does not yet exist any tool support (see Section~\ref{sect:discussion}), ii) it allows us to do user tests in future research, exploring the difference between GRL and RationalGRL, iii) tool support is an excellent verification to ensure our formal framework and algorithms actually work, iv) having a light-weight web-based version of GRL is useful for the community in general. In this section we briefly highlight some features of the tool, and we explain some of its current limitations.

GRL has a well-documented and well-maintained tool called jUCMNav~\cite{jUCMNav}\footnote{\url{http://jucmnav.softwareengineering.ca/foswiki/ProjetSEG}}, which is implemented as an Eclipse plugin. Although it is a rich tool with many features, it can take significant time to set it up properly and is quite feature-heavy. Therefore, we have decided to implement our own lightweight modelling tool as our proof-of-concept. The tool can be accessed from:

\begin{quote}
\url{http://www.rationalgrl.com}
\end{quote}

The RationalGRL tool is an open-source\footnote{\url{http://www.github.com/marcvanzee/RationalGRL}}, web-based Javascript application, which runs on all modern browsers. The tool is based on our formalization in Section~\ref{sect:formalframework}. Because this formalization is very close to the GRL metamodel, the models can be easily exported to jUCMNav.

When the tool starts, the user is presented with a screen as in Figure~\ref{fig:tool:overview}. This screen shows the palette of elements and links to the left, a canvas on which RationalGRL models can be built, and a pane with details of the currently selected elements on the canvas. Elements and links can be added to the canvas by selecting them on the left and clicking on the canvas, thus capturing the \textsf{INTRO} operation (Section~\ref{sect:formalframework:intro}). This makes it possible to build and modify regular GRL models using the tool. 

The details panel can be used to answer the critical questions of the RationalGRL framework (Table~\ref{table:argument-schemes}): Figure~\ref{fig:tool:overview}, for example, shows two critical questions for the goal element ``Generate cars'', namely ``Can the goal be realized?'' (CQ3) and ``Is the goal relevant/useful?'' (CQ11). Clicking the ``Answer'' button next to a critical question allows the user to answer the question. As an example, consider Figure~\ref{fig:tool:cqdetails}, where the softgoal ``Easy to use'' is questioned with the relevancy question (CQ11). It is possible to select the answer and provide an explanation for the answer. 

Assume that in the example of Figure~\ref{fig:tool:cqdetails} the user selects ``No'' and clicks the ``Answer question'' button. A new argument is then automatically created that attacks the softgoal and the details pane shows the critical question as answered (Figure~\ref{fig:tool:cqeffect}). This is an implementation of a \textsf{DISABLE} algorithm (Section~\ref{sect:formalframework:disable}) similar to Algorithm~\ref{alg:actor-not-relevant}: a new argument is added that attacks the existing argument. It is also possible to attack arguments by adding an ``Argument'' element and an ``Attack'' link manually. Consider, for example, Figure~\ref{fig:tool:argument}. Here, a new argument ``Necessary'', which attacks the previously generated argument based on CQ11, has been added by the user. As the details pane shows, this new argument is not based on a CQ. It is further worth noting that it is possible to provide further a explanation in argument elements, allowing for more fine-grained rationalizations. 

\begin{figure}[t]
\centering
\includegraphics[width=0.8\columnwidth]{img/tool/details_softgoal}
\caption{Critical question details pane for ``Is the softgoal relevant/useful?''}
\label{fig:tool:cqdetails}
\end{figure}

\begin{figure}[b]
\centering
\includegraphics[width=\columnwidth]{img/tool/attack_softgoal}
\caption{Effect of answering CQ11: ``Is the softgoal relevant/useful?'' with ``No''}
\label{fig:tool:cqeffect}
\end{figure}

The RationalGRL tool computes the acceptability of arguments on the fly. In the example of Figure~\ref{fig:tool:cqeffect}, the original (argument for) softgoal ``Easy to use'' is rejected because it's only attacker ``CQ11 - Redundant'' is accepted. However, if we then attack ``CQ11 - Redundant'' with a new argument ``Necessary'', the original argument for ``Easy to use'' is again accepted because its only attacker is rejected  (Figure~\ref{fig:tool:argument}). Note that when computing the acceptability of arguments, the RationalGRL tool makes the assumption that there are no attack cycles in the model and that hence there is one unique preferred extension (cf. Section~\ref{sect:formalframework:rationalgrl}) -- if the user creates an attack cycle, an error message is shown. 

\begin{figure}[t]
\centering
\includegraphics[width=0.8\columnwidth]{img/tool/reinstate_softgoal}
\caption{Attacking an argument with a generic argument ``Necessary''}
\label{fig:tool:argument}
\end{figure}

The final operation that has been captured in the RationalGRL tool is \textsf{REPLACE}. For usability purposes, this operation has been implemented differently than in our formal framework (Section~\ref{sect:formalframework:disable}). In the formal framework, a \textsf{REPLACE} introduces a new argument that replaces (and therefore attacks) all previous arguments with the same identifier. Including \textsf{REPLACE} like this in the tool would mean that, when the name of an element is changed, we would need to render all the previous versions of the element on the canvas of the tool. Because this would quickly become very cumbersome, we decided to implement a naming history for each element. For example, in Figure~\ref{fig:tool:overview}, the goal ``Generate cars'' was previously named ``Add cars''. In this way, a history is kept of how an element name was clarified without cluttering the model canvas. Similarly, the decomposition type can be changed in the details pane (Figure~\ref{fig:tool:overview}) without introducing new elements that attack the original ones (cf. Figure~\ref{fig:example-small3}, where the old decomposition type $AND$ of ``Generate traffic'' is shown as an explicit argument).  

The tool further implements the RationalGRL to GRL translation (Algorithm~\ref{alg:translation:to-grl}) through an export function. RationalGRL models built in the RationalGRL tool can be exported to the \texttt{.grl} file format, which can be imported in jUCMNav. Two examples of this export are provided in Appendix~\ref{sect:tool-screenshots}. Figure~\ref{fig:tool:figfrompaper} shows the model that was discussed earlier in this paper (Figure~\ref{fig:example-small3}) in the RationalGRL tool. Recall that translating this model to GRL provided us with the model in Figure~\ref{fig:example-small}. This is also what follows from our export: if we export the RationalGRL model and then import the resulting GRL model in jUCMNav, we get the model in Figure~\ref{fig:tool:figfrompaper1}, which is the same as Figure~\ref{fig:example-small}.

Our translation and export function uses the argument acceptability as a way of determining the GRL model. Take for example, the RationalGRL model in Figure~\ref{fig:tool:multipleattack}. A positive contribution is added from ``Keep same cars'' to ``Simple design''. More inportantly, an argument ``not enough CPU'' has been added to the model. This argument attacks the ``Realistic simulation'' softgoal and the ``Create new cars'' task, arguing that there is not enough processing power for either of these to be feasible in the traffic simulator. Now, if we export this to GRL, the GRL elements that are rejected or disabled (greyed out in Figure~\ref{fig:tool:multipleattack}) are not included. This can be seen in Figure~\ref{fig:tool:multipleattack1}, where the jUCMNav GRL model that was exported from the RationalGRL tool is shown. The pair of figures \ref{fig:tool:multipleattack} and \ref{fig:tool:multipleattack1} also nicely shows the added value of RationalGRL: Figure~\ref{fig:tool:multipleattack} shows that there can be a larger discussion and rationalization underlying even a fairly simple goal model such as the one in Figure~\ref{fig:tool:multipleattack1}. 

   

 
